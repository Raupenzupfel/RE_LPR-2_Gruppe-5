\section{Versuchsziele}
\begin{figure}[!h]
		\centering
		\includegraphics[width=0.7\textwidth]{Abbildungen/Thurow_Deckblatt}
		\caption{Animation des Versuchsaufbaus [Versuchsanleitung] }
		\label{fig:Versuchsaufbau}
\end{figure}

Mit Blick auf die Ziele der Bundesregierung den Endenergieverbrauch bis 2030 um 24\% zu senken gewinnt Energieeffizienz im Bausektor mit rasender Geschwindigkeit 
an Relevanz. Die Effizienzklassen für Neubauten legen den Fokus auf Dämmungen, wodurch in den nächsten Jahrzehnten viel Gas und Kohle eingespart werden kann. 
Der Laborversuch soll ein Verständnis für Dämmstoffe und die Klassifikation dieser durch den Wärmedurchgangskoeffizienten (U-Wert) schaffen.  
Der Versuchsaufbau besteht aus einem Modellhaus (\autoref{fig:Versuchsaufbau}) mit austauschbaren Wänden, einem abnehmbaren Dach und einer Glühlampe als Wärmequelle.
 Ziel ist es, aus den aufgenommenen Temperaturdifferenzen, in der Auswertung die Wärmeleitfähigkeit und den Wärmedurchgang der einzelnen Materialien zu bestimmen.\\\\
Aus Erfahrungswerten und bisherigen Vorlesungsveranstaltungen liegt die Hypothese nah, dass einige der Materialien Wärme schlechter leiten als andere. 
Während Polysterol, oftmals auch als Dämmstoff genutzt, eine schlechte Wärmeleitfähigkeit aufweisen wird, leitet das Floatglas Wärme wahrscheinlich wesentlich besser und 
führt somit auch zu einem größeren und damit schlechterem U-Wert. Erwartbar ist außerdem, dass die U-Werte in der zweiten Messung geringer sein werden als in der ersten Messung, da das Schichtholz durch 
eine weitere Dämmschicht ergänzt wurde.