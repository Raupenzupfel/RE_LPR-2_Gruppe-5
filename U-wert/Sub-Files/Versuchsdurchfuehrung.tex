\section{Versuchsdurchführung}

\begin{figure}[H]
    \centering
    \includegraphics[width=0.9\textwidth]{Abbildungen/WBK_UWERT.jpg}
    \caption{Aufnahmen der Wärmebildkamera des Versuchsaufbaus}
    \label{fig:230715_WBK}
\end{figure}

Zu Beginn des Versuchs werden die Temperaturfühler an die Thermometer angeschlossen und am Modellhaus angebracht. 
Es wird jeweils ein Temperaturfühler an der Außen- und Innenseite des Modellhauses befestigt um Außen- und Innenlufttemperatur messtechnisch zu erfassen. Im nächsten Schritt werden vier unterschiedlichen Wände (Schichtholz unterschiedlicher Dicke, Polystyrol und Glas) mit Temperaturfühlern an Innen- und Au"senseite ausgestattet. Im letzten Schritt wird 
das Modellhaus mit dem gedämmten Deckel verschlossen.

\begin{figure}[H]
    \centering
    \includegraphics[width=0.9\textwidth]{Abbildungen/M2_Modellhaus.jpeg}
    \caption{Versuchsaufbau der 2. Messreihe}
    \label{fig:230715_M2}
\end{figure}

Die Wärmequelle (Glühlampe), wird in Betrieb genommen und nach Erreichen des stationären Zustandes werden die Temperatur-Messreihen in 30 Sekunden Abständen aufgenommen. Nach 5 Minuten endet die Messreihe.
 Danach werden die beiden Holzwände um Polystyrol beziehungsweise Polystyrol und Messaufbau zwei zusätzlich um eine dazwischen liegende Luftschicht erweitert.
Im Anschluss werden erneut Messreihen aufgenommen. 
Abschließend werden mithilfe der Thermografiekamera (\autoref{fig:230715_WBK}) mögliche thermische Schwachstellen ermittelt.