\section{Auswertung}
\subsection{Bestimmung der Wärmeleitfähigkeiten und des Wärmedurchgangskoeffizienten für Polystyrol(1,84 cm), Holz (1,22 cm,  2,24 cm) und Glas(0,5 cm)}
Zunächst wird die Formel für die Wärmestromdichte benötigt. Diese lautet wie folgt:
\begin{equation}
\dot q=\frac{ Q }{ A }=h_{ i }\cdot(T_{ Li }-T_{Wi})
  \label{eq:230514_Wärmestromdichte}
\end{equation}
Der Wärmeübergangskoeffizient $h_{i}$ kann mithilfe des Wärmeübergangswiderstand $R_{si}$ aus DIN4108-4 ($R_{si}=0,13 m^2$) bestimmt werden.
\begin{equation}
h_{i}=\frac{ 1 }{ R_{ si } }=\frac{ 1 }{ 0,13m^2\cdot K \cdot W^{-1} }
  \label{eq:230514_Wärmeübergangskoeffizient}
\end{equation}
Unter der Annahme, dass die Wärmestromdichte von innen nach außen konstant ist, ist der innere Wärmestrom gleich dem gesamten Wärmestrom. Dank diesem Umstand lässt sich der U-Wert mithilfe von 3 gemessenen Temperaturen berechnen.
\begin{equation}
\frac{ T_{Li} - T_{Wi}}{ R_{ si } }=U \cdot T_{Li} - T_{La}
  \label{eq:230514_u1}
\end{equation}
nach U umgestellt ergibt sich: 
\begin{equation}
U = \frac{ T_{Li} - T_{Wi}}{ R_{ si } \cdot  (T_{Li} - T_{La})}
  \label{eq:230514_u2}
\end{equation}
Der Wärmeübergangskoeffizient für außen $h_{e}$ lässt sich mithilfe des spezifischen Wärmestroms bestimmen. Dabei ist $R_{se}=0,04 \frac{ m^2 \cdot K }{W}$ (siehe DIN4108-4)
\begin{equation}
h_{e}=\frac{ q }{ T_{Wa} - T_{La} }   \qquad  mit     \qquad       R_{se}=\frac{ 1 }{ h_{e} }
  \label{eq:230514_h_e}
\end{equation}
Anschließend lässt sich die Wärmeleitfähigkeit wie folgt berechnen:
\begin{equation}
\lambda=\frac{ \dot q \cdot d }{  ( T_{Wi} - T_{Wa} )}
  \label{eq:230514_lamda}
\end{equation}
\begin{table}[!ht]
    \centering
\caption{Berechnete Werte Für die Wärmestromdichte $\dot q$, den Wärmedurchgangskoeffizienten U und die Wärmeleitfähigkeit  $\lambda$ }
\label{tab:230524_Messdaten_Messreihe1(1)}
    \begin{tabular}{|l|l|l|}
    \hline
        Wärmestromdichte $\dot q$ in W/$m^2$ & Wärmedurchgangskoeffizient (U) in W/($m^2$*K) & Wärmeleitfähigkeit ($\lambda$) in W/(mK) \\ \hline
        Polystyrol 1,84 cm & 45,52 & 1,35 \\ \hline
        Holz 1,22 cm & 114,48 & 3,40 \\ \hline
        Holz 2,24 cm & 103,14 & 3,06 \\ \hline
        Glas 0,5 cm & 122,31 & 3,63 \\ \hline
    \end{tabular}
\end{table}
In Tabelle \textcolor{red}{..........}  sind dabei die berechneten Werte dargestellt.
%
%
% 
%
%
\subsection{Berechnen Sie Wärmedurchlasswiderstand R für den zweiten Wandaufbau der 2. Messreihe}
Der Wärmewiderstand lässt sich mit folgender Formel berechnen:
%
\begin{equation}
R=\frac{(T_{\textcolor{red}{!!!!!!!!!!!!!!!!! noch bennenen }} \cdot T_{\textcolor{red}{!!!!!!!!!!!!!}} )}{\dot q}
  \label{eq:230522_Wärmewiderstand}
\end{equation}
%
In Tabelle \textcolor{red}{..........}  ist dabei der Wärmewiderstand der verschiedenen Schichten zu sehen.

\textcolor{red}{\textbf{Noch zu tun Fehler abschätzen. Und Fotos einfügen, sowie  }}