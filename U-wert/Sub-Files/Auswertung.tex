\section{Auswertung}
\subsection{Bestimmung der Wärmeleitfähigkeiten und des Wärmedurchgangskoeffizienten für Polystyrol(1,84 cm), Holz (1,22 cm, 3,21 cm) und  Glas(0,5 cm)}
Zunächst wird die Formel für die Wärmestromdichte benötigt. Diese lautet wie folgt:
\begin{equation}
\dot q=\frac{ Q }{ A }=h_{ i }\cdot(T_{ Li }-T_{Wi })
  \label{eq:230514_Wärmestromdichte}
\end{equation}
Die Fläche beträgt dabei ............... Der Wärmeübergangskoeffizient $h_{i}$ kann mit Hilfe des Wärmeübergangswiderstand $R_{si}$ aus DIN4108-4 ($R_{si}=0,13 m^2$) bestimmt werden.
\begin{equation}
hi=\frac{ 1 }{ R_{ si } }=\frac{ 1 }{ 0,13m^2\cdot K \cdot W^{-1} }
  \label{eq:230514_Wärmeübergangskoeffizient}
\end{equation}
Unter der Annahme, dass die Wärmestromdichte von innen nach aussen konstant ist, ist der innere Wärmestrom gleich dem gesammten Wärmestrom. Dank diesem Umstand lässt sich der U-Wert mit Hilfe von 3 gemessenen Temperaturen berechnen.
\begin{equation}
\frac{ T_{Li} - T_{Wi}}{ R_{ si } }=U \cdot T_{Li} - T_{La}
  \label{eq:230514_u1}
\end{equation}
nach U umgestellt ergibt sich: 
\begin{equation}
U = \frac{ T_{Li} - T_{Wi}}{ R_{ si } \cdot  T_{Li} - T_{La}}
  \label{eq:230514_u2}
\end{equation}
Der Wärmeübergangskoeffizient für außen $h_{e}$ lässt sich mit Hilfe des spezifischen Wärmestroms bestimmen. 
\begin{equation}
h_{e}=\frac{ q }{ T_{Wa} - T_{La} }   \qquad  mit     \qquad       R_{se}=\frac{ 1 }{ h_{e} }
  \label{eq:230514_h_e}
\end{equation}
Anschließend lässt sich die Wärmeleitfähigkeit wie folgt berechnen:
\begin{equation}
\lambda=\frac{ Q \cdot d }{ A \cdot ( T_{Wi} - T_{Wa} )}
  \label{eq:230514_lamda}
\end{equation}
In Tabelle ..........  sind dabei die berechneten Werte dargstellt.
\subsection{Berechnen Sie Wärmedurchlasswiderstand R für den zweiten Wandaufbau der 2. Messreihe}
Der Wärmewiderstand für den inneren- $R_{si}=0,13\frac{ m^2 \cdot K }{W} $ und den äüßeren- $R_{se}=0,04 \frac{ m^2 \cdot K }{W}$ Wärmewiderstand sind dabei in DIN 4108-4 festgelegt.