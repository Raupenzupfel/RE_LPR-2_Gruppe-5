\section{Theoretischer Hintergrund}
Die wesentlichen Versuchsgrundlagen (gültige Gesetze, Gleichungen, Axiome etc.) sind in zusammenhängenden selbst formulierten Sätzen kurz (je nach Versuch ca. 2 bis 3 Seiten) darzustellen. 
Die im Text eingearbeiteten Gleichungen sind mit Nummern in runden Klam-mern auf der rechten Seite fortlaufend zu nummerieren, wie in \mref{Fl.Kreis} dargestellt. 
Weitere Informationen zur nutzung der Mathematik Umgebung in LaTeX sind im Internet zu finden.
%
\begin{equation}
	A_{Kreis} = \pi r^{2}
	\label{eq:Fl.Kreis}
\end{equation}
%
Die dabei verwendeten Abkürzungen sind, entweder am Anfang in einem Abkürzungs- und Symbolverzeichnis, oder direkt nach der ersten Verwendung der Abkürzung zu erklären!\\