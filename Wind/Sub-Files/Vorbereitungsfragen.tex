\section{Vorbereitungsfragen}
\subsection{Der Leistungsbeiwert \texorpdfstring{$c_P$}{} in der Praxis}
Der Leistungsbeiwert $c_P$ ist der tatsächliche Anteil der Energie des Windes, der in mechanische Energie umgesetzt werden kann und beschreibt den Wirkungsgrad einer Wind-Energie-Anlage. Er wird nach \autoref{eq:230609_1531-Leistungsbeiwert_cp} oder \autoref{eq:230609_1532-Leistungsbeiwert_cp} bestimmt und ist, wie in \ref{eq:230609_1532-Leistungsbeiwert_cp} erkennbar von den anlagenspezifischen Eigenschaften beeinflusst.



\subsection{Herleitung des theoretisch maximalen Leistungsbeiwerts \texorpdfstring{$c_P$}{} }
Der maximale Leistungsbeiwert, auch Betz'scher Leistungsbeiwert von knapp 60\%, wird wie folgt bestimmt.

\begin{equation}
P_{WKA}= \frac{1}{2} \cdot \rho \cdot A \cdot v_2 \cdot (v_1^2 - v_3^2)
\label{eq:P_WKA} 
\end{equation}

\begin{equation}
v_2= \frac{v_1+v_3}{2} 
\label{eq:v_2} 
\end{equation}

In \autoref{eq:P_WKA} wird durch \autoref{eq:v_2} $v_2$ eliminiert, wodurch diese dann klar als die \color{green}{Leistung des Windes}\color{black}{} und \color{blue}{Leistungsbeiwert} \color{black}{} getrennt werden kann. 

\begin{equation}
P_{WKA}= \color{green}{\frac{1}{2} \cdot \rho \cdot A \cdot v_1^3 } \cdot \color{blue}{\frac{1}{2} (1-\frac{v_3^2}{v_1^2} + \frac{v_3}{v_1} - \frac{v_3^3}{v_1^3}})\color{black}{}
\label{eq:P_WKA2} 
\end{equation}



Unter der Annahme das $x=\frac{v_3}{v_1}$ entspricht, entsteht eine einfache quadratische Gleichung (\ref{eq:CPMAX}). Diese wird abgeleitet um Lokale Extrempunkte zu bestimmen (\autoref{eq:CPMAX2}). 

\begin{equation}
c_{P,MAX}= \frac{1}{2} \cdot (1-x^2+x-x^3)
\label{eq:CPMAX}
\end{equation}

\begin{equation}
c_{P,MAX}=\frac{1}{2} \cdot (-3x^2-2x+1)
\label{eq:CPMAX2}
\end{equation}
 
Der positive Extrempunkt ist der Maximalwert von $c_P$.  $$x_{1/2}= \frac{1}{3}, -1$$
$$c_{P,MAX}(\frac{1}{3})= \frac{16}{27} = 0,59 = 59\%$$

\subsection{Abschätzungen des Schubbeiwerts \texorpdfstring{$c_S$}{} im Stillstand und bei
Leerlaufdrehzahl}

Der Schubbeiwert steigt mit der Schnelllaufzahl $\lambda$ stetig an (\ref{fig:cszulambda}). Die schmalen Flügel der Windkraftanlage lassen im Stillstand den Wind fast ungehindert durch die Rotorebene strömen weshalb auch der Schubbeiwert sehr klein ist. Steigt die Schnellaufzahl steigt auch der Schubbeiwert und erreicht im Leerlauf sein Maximum was bei ungefähr 1,25 liegt und dem Widerstandsbeiwert einer geschlossenen Kreisscheibe entspricht. 


\subsection{Der Momentenbeiwert \texorpdfstring{$c_M$}{} in der Praxis}

Der Momentenbeiwert beschreibt die aerodynamischen Kräfte, die auf den Rotor der Windkraftanlage wirken und ein Drehmoment erzeugen. Er wird zur Auslegung der Rotorblätter genutzt und beeinflusst Geometrie und Profilform. Genau wie der Schubbeiwert ist er wichtig für die Sicherheit und Stabilität der Anlage. 


\subsection{Der Schubbeiwert \texorpdfstring{$c_S$}{} in der Praxis}

Der Schubbeiwert einer Windkraftanlage ist relevant für die Leistungsberechnung, Auslegung der Anlagen, Steuerung und Regelung als auch für die Lastberechnung und strukturelle Integrität einer Windkraftanlage. 

\subsection{Herleitung des Schubbeiwerts \texorpdfstring{$c_S$}{} bei Auslegungsschnelllaufzahl \texorpdfstring{$\lambda$}{}}

Für den Schubbeiwert im Auslegungspunkt wird die Gleichung für die Leistung der WKA (\autoref{eq:P_WKA3}) mit der Gleichung für den Schubbeiwert (\autoref{eq:FS}) im Zusammenhang mit der Leistung in der Rotorebene (\autoref{eq:P_R})  und der Geschwindigkeit in der Rotorebene nach Betz (\autoref{eq:v_R}) gleichgesetzt. 

\begin{equation}
P_{WKA}= \frac{1}{2} \cdot \rho \cdot A \cdot v_1^3  \cdot c_P
\label{eq:P_WKA3} 
\end{equation}

\color{green}{}
\begin{equation}
F= \frac{1}{2} \cdot \rho \cdot A \cdot v_1^2  \cdot c_S
\label{eq:FS} 
\end{equation}
\color{black}{}


\begin{equation}
P_R= F \cdot v_2 
\label{eq:P_R} 
\end{equation}

\color{blue}{}
\begin{equation}
v_R= \frac{2}{3} \cdot v_1
\label{eq:v_R} 
\end{equation}
\color{black}{}


\autoref{eq:FS} und \autoref{eq:v_R} werden in \autoref{eq:P_R} eingesetzt und mit \autoref{eq:P_WKA3} gleichgesetzt. 

\begin{equation}
\frac{1}{2} \cdot \rho \cdot A \cdot v_1^3 \cdot c_P = \color{green}{}\frac{1}{2} \cdot \rho \cdot A \cdot v_1^2  \cdot c_S \cdot \color{blue}{}\frac{2}{3} \cdot v_1 \color{black}{}
\label{eq:c_s1} 
\end{equation}

Gekürzt und umgestellt ergibt sich für $c_S$:

$$c_S= \frac{2}{3} \cdot c_P = \frac{2}{3} \cdot \frac{16}{27} = \frac{8}{9}$$


Der Schubbeiwert $c_S$ liegt sowohl für Schnell- als auch für Langsamläufer im Auslegungspunkt bei $\frac{8}{9}$.

\label{sec:Vorbereitungsfragen}
