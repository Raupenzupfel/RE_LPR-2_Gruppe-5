\section{Vorbereitungsfragen}
\subsection{Wozu wird der Leistungsbeiwert \texorpdfstring{$c_P$}{} in der Praxis benötigt?}
Der Leistungsbeiwert $c_P$ ist der tatsächliche Anteil der Energie des Windes, der in mechanische Energie umgesetzt werden kann und beschreibt den Wirkungsgrad einer Wind-Energie-Anlage. Er wird nach \ref{eq:Leistungsbeiwert_cp} oder \ref{eq:Leistungsbeiwert_cp2} bestimmt und ist, wie in \ref{eq:Leistungsbeiwert_cp2} erkennbar von den anlagenspezifischen Eigenschaften beeinflusst.



\subsection{Leiten Sie den theoretisch maximalen Wert des Leistungsbeiwerts \texorpdfstring{$c_P$}{} her.}
Der maximale Leistungsbeiwert, auch Betz'scher Leistungsbeiwert, von knapp 60\% wird wie folgt bestimmt.

\begin{equation}
P_{WKA}= \frac{1}{2} \cdot \rho \cdot A \cdot v_2 \cdot (v_1^2 - v_3^2)
\label{eq:P_WKA} 
\end{equation}

\begin{equation}
v_2= \frac{v_1+v_3}{2} 
\label{eq:v_2} 
\end{equation}

In \ref{eq:P_WKA} wird durch \ref{eq:v_2} $v_2$ eliminiert, wodurch diese dann klar als die \color{green}{Leistung des Windes} mit dem \color{blue}{Leistungsbeiwert} erkennbar ist. 

\begin{equation}
P_{WKA}= \color{green}{\frac{1}{2} \cdot \rho \cdot A \cdot v_1^3 } \cdot \color{blue}{\frac{1}{2} (1-\frac{v_3^2}{v_1^2} + \frac{v_3}{v_1} - \frac{v_3^3}{v_1^3}})
\label{eq:P_WKA2} 
\end{equation}

\color{black}{}

Unter der Annahme das $x=\frac{v_3}{v_1}$ entspricht, entsteht eine einfache Quadratische Gleichung (\ref{eq:CPMAX}). Diese Wird abgeleitet um Lokale Extrempunkte zu bestimmen (\ref{eq:CPMAX2}). 

\begin{equation}
c_{P,MAX}= \frac{1}{2} \cdot (1-x^2+x-x^3)
\label{eq:CPMAX}
\end{equation}

\begin{equation}
c_{P,MAX}=\frac{1}{2} \cdot (-3x^2-2x+1)
\label{eq:CPMAX2}
\end{equation}
 
Die Extrempunkte  $x_{1/2}= \frac{1}{3}, -1$
$c_{P,MAX}(\frac{1}{3})=0,59 = 59\%$

\subsection{Geben Sie Abschätzungen für den Schubbeiwert \texorpdfstring{$c_S$}{} im Stillstand und bei
Leerlaufdrehzahl an und begründen Sie diese.}



\subsection{Wozu wird der Momentenbeiwert \texorpdfstring{$c_M$}{} in der Praxis benötigt?}



\subsection{Wozu wird der Schubbeiwert \texorpdfstring{$c_S$}{} in der Praxis benötigt?}



\subsection{Leiten Sie den Wert des Schubbeiwerts \texorpdfstring{$c_S$}{} für die Auslegungsschnelllaufzahl \texorpdfstring{$\lambda$}{}
her.}






\label{sec:Vorbereitungsfragen}
