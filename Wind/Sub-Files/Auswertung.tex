\section{Auswertung}
\label{sec:Auswertung}
\subsection{Anlaufverhalten}
In diesem Szenario wird der Pitchwinkel schrittweise variiert, und für jede Servo-Einstellung  wird das abgelesene Moment aufgezeichnet. Anschließend wird der Pitchwinkel sowohl durch Kalibrierung als auch unter Verwendung des Momentenbeiwerts berechnet.
\begin{table}[ht!]
    \centering
    \caption{Messwerte}
    \label{tab_Messwerte_Anlauf_230615}
    \begin{tabular}{|l|l|l|l|l|l|}
        \hline
        \rowcolor[HTML]{70AD47} 
        {\color[HTML]{FFFFFF} \textbf{v\_Wind   Soll}} & {\color[HTML]{FFFFFF} \textbf{v\_Wind Ist}} & {\color[HTML]{FFFFFF} \textbf{Pitch abgelesen}} & {\color[HTML]{FFFFFF} \textbf{Pitch erwartet}} & {\color[HTML]{FFFFFF} \textbf{C\_M,0}} & {\color[HTML]{FFFFFF} \textbf{C\_M,1}} \\ \hline
        \rowcolor[HTML]{70AD47} 
        in m/s                                         & in m/s                                      & in °                                            & in °                                           & -                                      & -                                      \\ \hline
        \rowcolor[HTML]{E2EFDA} 
        1                                              & 1                                           & -                                               & -                                              & 0,666                                  & 0,666                                  \\ \hline
        \rowcolor[HTML]{C6E0B4} 
        1,5                                            & 1,6                                         & 45                                              & -                                              & 0,296                                  & 0,260                                  \\ \hline
        \rowcolor[HTML]{E2EFDA} 
        2                                              & 2,1                                         & 32                                              & -                                              & 0,167                                  & 0,151                                  \\ \hline
        \rowcolor[HTML]{C6E0B4} 
        2,5                                            & 2,45                                        & -5                                              & 5 bis 25                                       & 0,107                                  & 0,111                                  \\ \hline
    \end{tabular}
\end{table}


\subsection{Leerlauf und Maximale Schnelllaufzahl}
Im Rahmen der Untersuchungen wurden bei verschiedenen konstanten Windgeschwindigkeiten verschiedene Messungen durchgeführt. Die Auswertung dieser Messungen beinhaltet mehrere wichtige Parameter, die für jede der untersuchten Windgeschwindigkeiten dargestellt werden. Die entsprechenden Werte sind in der folgenden Tabelle zusammengefasst:

\begin{table}[ht!]
    \centering
    \caption{Messwerte Leerlaufverhalten und maximale Schnelllaufzahl}
    \label{tab_Messwerte_Leerlaufverhalten}
    \small
    \begin{tabular}{|l|l|l|l|l|l|}
    \hline
    \rowcolor[HTML]{70AD47} 
    {\color[HTML]{FFFFFF} \textbf{Pitch}} & {\color[HTML]{FFFFFF} \textbf{Pitch2}} & {\color[HTML]{FFFFFF} \textbf{Generator Spannung}} & {\color[HTML]{FFFFFF} \textbf{Drehzahl}} & {\color[HTML]{FFFFFF} \textbf{Blattspitzen geschw.}} & {\color[HTML]{FFFFFF} \textbf{Schnellaufzahl}} \\ \hline
    \rowcolor[HTML]{70AD47} 
    soll                                  & berechnet                              & abgelesen                                          & abgelesen                                & berechnet                                            & berechnet                                      \\ \hline
    \rowcolor[HTML]{70AD47} 
    in °                                  & in °                                   & in V                                               & in min-1                                 & in m/s                                               & -                                              \\ \hline
    \rowcolor[HTML]{C6E0B4} 
    -5                                    & -3,51                                  & 8,2                                                & 2068,7                                   & 36,83                                          & 5,67                                    \\ \hline
    \rowcolor[HTML]{E2EFDA} 
    0                                     & -1,34                                  & 9                                                  & 2272                                     & 40,45                                          & 6,22                                    \\ \hline
    \rowcolor[HTML]{C6E0B4} 
    5                                     & 5,45                                   & 9,48                                               & 2390,9                                   & 42,56                                          & 6,55                                    \\ \hline
    \rowcolor[HTML]{E2EFDA} 
    10                                    & 12,17                                  & 9,9                                                & 2512,5                                   & 44,73                                          & 6,88                                    \\ \hline
    \rowcolor[HTML]{C6E0B4} 
    15                                    & 18,01                                  & 9,78                                               & 2476,1                                   & 44,08                                          & 6,78                                    \\ \hline
    \rowcolor[HTML]{E2EFDA} 
    20                                    & 23,24                                  & 9,2                                                & 2344,8                                   & 41,74                                          & 6,42                                     \\ \hline
    \rowcolor[HTML]{C6E0B4} 
    25                                    & 28,22                                  & 8,6                                                & 2187,5                                   & 38,94                                          & 5,99                                    \\ \hline
    \rowcolor[HTML]{E2EFDA} 
    30                                    & 32,81                                  & 8                                                  & 2022,2                                   & 36,00                                           & 5,54                                     \\ \hline
    \rowcolor[HTML]{C6E0B4} 
    35                                    & 37,10                                 & 7,3                                                & 1863,1                                   & 33,17                                          & 5,10                                    \\ \hline
    \rowcolor[HTML]{E2EFDA} 
    40                                    & 40,97                                  & 6,85                                               & 1739,5                                   & 30,97                                          & 4,76                                    \\ \hline
    \rowcolor[HTML]{C6E0B4} 
    45                                    & 44,98                                  & 6,6                                                & 1670,8                                   & 29,74                                         & 4,58                                    \\ \hline
    \rowcolor[HTML]{E2EFDA} 
    50                                    & 48,87                                  & 6,2                                                & 1573,1                                   & 28,00                                         & 4,31                                    \\ \hline
    \end{tabular}
    \end{table}
     Dabei wurden die Pitch-Einstellung bzw. der rechnerisch korrigierte Pitch-Winkel, um die Neigung der Rotorblätter und deren 
     präzise Ausrichtung zur Anströmung des Windes zu ermitteln, in der Tabelle dargestellt. Ebenfalls wurde die Generator-Leerlaufspannung $U_{Gen_0}$ erfasst, 
     um die elektrischen Eigenschaften des Generators im Leerlauf zu untersuchen. Ein weiterer relevanter Wert ist die Leerlaufdrehzahl des Rotors $n_0$, 
     die Auskunft über die Rotor-Drehzahl bei fehlender Last gibt. Zudem wurde die Blattspitzengeschwindigkeit des Rotors $u_{tip}$ ermittelt, um die Geschwindigkeit an den äußersten Enden der Rotorblätter zu bestimmen. 
     Die Schnelllaufzahl $\lambda$, als Verhältnis zwischen der relativen Windgeschwindigkeit und der Blattspitzengeschwindigkeit, wurde untersucht, um die aerodynamische Effizienz des Systems zu bewerten. Besonders wichtig war die Vermerkung der maximalen
    Schnelllaufzahl $\lambda_{max}$ für jede Windgeschwindigkeit $v_{Wind}$, um die optimale Auslastung des Rotors bei verschiedenen Windbedingungen zu erkennen. Die Auswertung dieser Parameter lieferte
    wertvolle Erkenntnisse über das Verhalten des Windkraftsystems bei unterschiedlichen Windgeschwindigkeiten und ermöglichte die gezielte Verbesserung der Leistungsfähigkeit und Effizienz der Anlage. Diese Erkenntnisse tragen zur Weiterentwicklung und 
    Optimierung von Windenergieanlagen bei,
     um einen effizienten und nachhaltigen 
     Beitrag zur Energieerzeugung aus erneuerbaren Quellen zu gewährleisten.

\begin{equation}
    \centering
    u = 2 \pi \cdot r_{Rotor} \cdot \frac{n}{60 \frac{s}{min}}
    \label{eq:230615_Umfangsgeschwindigkeit}
\end{equation}

$$    u = 2 \pi \cdot 0,17m \cdot \frac{2512,5 \frac{1}{min}}{60 \frac{s}{min}}\approx 44,73\frac{m}{s} $$

\begin{equation}
    \centering
    \lambda = \frac{u}{v_1}
    \label{eq:230615_Schnelllaufzahl}
\end{equation}

$$\lambda = \frac{44,73}{6,5 \frac{m}{s}}\approx 6,88$$


\subsection{Dimensionslose Kennzahlen}

\begin{equation}
    \centering
    P_{Wind} = \frac{\rho}{2} \cdot \pi \cdot r_{Rotor}^2 \cdot v_1^3
    \label{eq:230615_Leistung_Wind}
\end{equation}

$$P_{Wind} = \frac{1,2 \frac{kg}{m^3}}{2} \cdot \pi \cdot (0,17 m)^2 \cdot (6,43)^3 \approx 14,46 W$$

\begin{equation}
    \centering
    M_{Gen} = 33,931\frac{mNm}{mV} \cdot U_{Rotor} + 3,1519 mNm - M_{Reibung}
    \label{eq:230615_Genratormoment}
\end{equation}

$$M_{Gen} = 33,931\frac{mNm}{mV} \cdot 2,1631 + 3,1519 mNm - 6,3mNm \approx 70,25 mNm$$

\begin{equation}
    \centering
    P_{Mech} = M_{Gen} \cdot 2 \pi \frac{n}{60\frac{s}{min}}
    \label{eq:230615_Mechanischeleistung}
\end{equation}

$$P_{Mech} = 70,25 \cdot 2 \pi \frac{1318,4 \frac{1}{min}}{60\frac{s}{min}} \approx 9,7W$$

\begin{equation}
    \centering
    c_M = \frac{M_{Gen}}{ \frac{\rho}{2} \cdot \pi \cdot r_{Rotor}^3 \cdot v_1^2}
    \label{eq:230615_Momentenbeiwert}
\end{equation}

$$c_M = \frac{70,25}{ \frac{1,2 \frac{kg}{m^3}}{2} \cdot \pi \cdot (0,17 m)^3 \cdot (6,43)^2}\approx 0,184$$

\begin{equation}
    \centering
    P_{El} = U \cdot I
    \label{eq:230615_Elektrische_Leistung}
\end{equation}

$$P_{El} = 3V \cdot 0,915A \approx 2,745W$$

% \begin{equation}
%     \centering
%     c_P = c_M \cdot \lambda
%     \label{eq:230615_Leistungsbeiwert}
% \end{equation}

% $$c_P = c_M \cdot \lambda$$

\begin{equation}
    \centering
    \eta_{Gen} = \frac{P_{El}}{P_{Mech}}
    \label{eq:230615_Generatorwirkungsgrad}
\end{equation}

$$\eta_{Gen} = \frac{2,745W}{9,7W} \approx 0,2833 \approx 28,33\% $$

\begin{equation}
    \centering
    \eta_{Gesamt} = \frac{P_{El}}{P_{Wind}}
    \label{eq:230615_Gesamtwirkungsgrad}
\end{equation}

$$\eta_{Gesamt} = \frac{2,745}{14,46}\approx 0,1898 = 18,98\%$$

\begin{equation}
    \centering
    F_S = F_{S,Mess} \cdot \frac{l}{h_N} \cdot f_{Korr}
    \label{eq:230615_Schubkraft}
\end{equation}

$$F_S = 0,8853N \cdot \frac{0,467m}{0,73m} \cdot 3,267 = 1,85N$$

\begin{equation}
    \centering
    C_S = \frac{F_S}{\frac{\rho}{2} \cdot \pi \cdot r_{Rotor}^2 \cdot v_1^2}
    \label{eq:230615_Schubbeiwert}
\end{equation}

$$C_S = \frac{1,85N}{\frac{1,2 \frac{kg}{m^3}}{2} \cdot \pi \cdot (0,17 m)^2 \cdot v_1^2} = 0,822$$
