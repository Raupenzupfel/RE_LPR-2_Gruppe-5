\section{Versuchsdurchführung}
\label{section:Versuchsdurchführung}
Der erste Teil des Versuchs untersucht das Anlaufverhalten der Windkraftanlage bei unterschiedlichen Windgeschwindigkeiten.
Es sind dementsprechend Windgeschwindigkeiten von 1,5 $\frac{m}{s}$, 2,0 $\frac{m}{s}$, 2,5 $\frac{m}{s}$ und 3,0 $\frac{m}{s}$ einzustellen.
Bei jeder dieser Windgeschwindigkeiten ist der Pitchwinkel, beginnend von -10°, so weit in der positven Bereich zu regeln, bis der Rotor zu drehen beginnt, der zu diesem Zeitpunkt angezeigte Pitchwinkel ist in der beigefügten Exceltabelle "'Windversuch"' im Tabellenblatt "'Anlaufverhalten"' notiert worden.\\
Im zweiten Versuchsabschnitt ist das Leerlaufverhalten und die maximale Schnelllaufzahl zu betrachten.
Hierfür ist der Generator in den Leerlauf geschaltet und der Windkanal auf konstant 6 $\frac{m}{s}$ eingestellt.
Der Pitchwinkel ist von -5° bis auf 50° in 5°-Schritten zu erhöhen.
Es sind für jeden Messpunkt die Generatorspannung und die Rotordrehzahl abzulesen.
Die Ergebnisse wurden in der bereits erwähnten Tabellenkalkulation im Tabellenblatt "'Leerlaufverhalten'" zusammengetragen.\\
Die dritte und letzte Messung dient der Ermittlung der dimensionslosen Kennzahlen.
Es sind ein fester Pitchwinkel von 10° und eine Windgeschwindigkeit von 6,5 $\frac{m}{s}$ einzustellen.
An den entsprechenden Messpunkten werden mittels Multimetern die Generatorspannung und der Generatorstrom gemessen.
Für die Messung der Windgeschwindigkeit, der Rotordrehzahl und des Drehmoments sind die in \autoref{section:Versuchsbeschreibung} beschriebenden Messgeräte zu verwenden.
Die Messpunkte sind anhand der elektronischen Last gegeben, so ist die Last in 5\%-Schritten, beginnend von 0\%, soweit zu erhöhen,bis bei der Messung der Generatorspannung über mehrere Messpunkte keine relevante Änderung erkennbar ist.
Die abzulesenden Werte sind abschließend im Tabellenblatt "'Dimensionslose Kennzahlen'" der beigefügten Tabellenkalkulation zusammengetragen.