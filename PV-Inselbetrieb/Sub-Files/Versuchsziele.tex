\section{Versuchsziele}
Der Versuch 'Photovoltaik-Inselbetrieb' beleuchtet zum tieferen Verständnis das Anlagenkonzept eines Photovoltaik-Inselsystems . Das System wird sowohl als System 
 für DC-Verbraucher als auch für AC-Verbraucher betrachtet. Das
 Verstehen und Aufstellen von Energiebilanzen wird geschult.\\
 Die Relevanz von Photovoltaik-Inselanlagen ist die vollständige Unabhängigkeit der Systeme. Die damit verbundenen Möglichkeiten reichen von der Versorgung kleinerer elektrischer Geräte wie Taschenrechner, Uhren oder GPS-Tracker bis hin zu der Möglichkeit eine Stromversorgung mit eigenem Netz für größere Geräte herzustellen.
 Mögliche Anwendungsbereiche sind hier die Versorgung von Geräten an Standorten
 an denen ein Netzanschluss nicht möglich ist, z.B. auf einer Bohr- oder Forschungsinsel
 oder dort wo noch kein Netz existiert und eine Inselanlage den ersten sicheren
 Stromversorger darstellt, wie es in einigen Entwicklungsländern bereits der Fall ist.\\
 %