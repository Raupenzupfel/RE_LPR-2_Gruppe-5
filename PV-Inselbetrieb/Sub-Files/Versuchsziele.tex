\section{Versuchsziele}
Der Versuch 'Photovoltaik-Inselbetrieb' wurde durchgeführt und
 soll im Folgenden ausgewertet werden. Die Durchführung und Auswertung des 
 Versuches verfolgt das Ziel die Komplexität eines solchen 
 Photovoltaik-Inselsystems zu verstehen. Dabei wird das System sowohl als System 
 für DC-Verbraucher als auch für AC-Verbraucher betrachtet. Des Weiteren ist das
 Verstehen und aufstellen von Energiebilanzen ein entscheidener Teil der Betrachtung.\\
 Die Relevanz von Photovoltaik-Inselanlagen und damit der Grund für die Durchführung
 dieses Versuches liegt in der vollständigen Unabhängigkeit der Systeme. Die damit
 verbundenen Möglichkeiten reichen von der Versorgung kleinerer elektrischer Geräte
 wie Taschenrechnern, Uhren oder GPS-Trackern bis hin zur Möglichkeit eine Stromversorgung
 inklusive Netz für größere Geräte herzustellen.
 Mögliche Anwendungsbereiche sind hier die Versorgung von Geräten an Standorten
 an denen ein Netzausbau nicht möglich ist, z.B. auf einer Bohr- oder Forschungsinsel
 oder dort wo noch kein Netz vorhanden ist und eine Inselanlage den ersten sicheren
 Stromversorger darstellt, wie in einigen Entwicklungsländern bereits der Fall.\\
 %