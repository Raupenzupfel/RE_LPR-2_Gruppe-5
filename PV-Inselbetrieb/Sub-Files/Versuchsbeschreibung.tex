\section{Versuchsbeschreibung}
\label{section:Versuchsbeschreibung}
Der Versuchsaufbau für diese Messungen besteht aus zwei eigenständigen Systemen.
Jedes der Systeme wird über eine eigene PV-Anlage versorgt, hierbei ist Anlage 1 nachgeführt und Anlage 2 stationär. Beide PV-Anlagen sind auf
der Dachterasse des G-Gebäudes der HTW Berlin am Campus Wilhelminenhof aufgeständert.
In \ref*{fig:AB1} und \ref*{fig:AB2} sind die zwei unterschiedlichen PV-Generatoren
zu sehen. Die weiteren Komponenten sind innerhalb des Gebäudes angebracht.\\
%
\begin{figure}[H]
	\centering
	\begin{minipage}{0.49\textwidth}
		\centering
		\includegraphics[width=1\textwidth]{Abbildungen/nachgeführte PV.jpeg}
		\caption{Pnachgeführter PV-Generator}	
		\label{fig:AB1}
	\end{minipage}
	\hfill
\begin{minipage}{0.49\textwidth}
	\centering
	\includegraphics[width=1\textwidth]{Abbildungen/stationäre PV.jpeg}
	\caption{stationärer PV-Generator}
	\label{fig:AB2}
\end{minipage}
\end{figure}
%
Hinsichtlich der Bauteile unterscheiden sich die Anlagen ebenfalls, für Aufbau 1 
wurden die folgenden Komponenten genutzt:
\begin{itemize}
	\item 8x Solarmodule SunWare SW-8046 
	\item 1x Wechselrichter SMA SUNNY ISLAND 2012/2224 
	\item 1x Laderegler SMA SUNNY ISLAND CHARGER 50 
	\item 1x Batteriesystem aus Blei-Gel-Batterien BAE 6V4PVV280
	\item 1x Ohmsche-induktive Last (hier:stufenweise regelbarer Staubsauger)
	\item 1x Ohmsche Last (hier: S-Bahn Heizung)
\end{itemize}

Für Aufbau 2 wurden folgende Komponenten genutzt:
\begin{itemize}
	\item 8x Solarmodule SunWare SW-8046 
	\item 1x verstellbarer Widerstand
\end{itemize}
