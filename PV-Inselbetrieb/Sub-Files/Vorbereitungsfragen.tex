\section{Vorbereitungsfragen}
Die Vorbereitungsfragen sollen aus Vorbereitung auf dem Versuch beantwortet werden. Ob diese auch als Teil des Protokolls erforderlich sind wird im jeweiligen Laborskript beschrieben.

\subsection{Photovoltaik-Inselanlage}
\subsubsection{Geben Sie vier Konzepte von PV-Inselanlagen mit je einem Einsatzbeispiel an}
\subsubsection{Die Anlagenkomponenten sollen im Inselsystem auf ihre Funktionsfähigkeit hin geprüft werden. Was messen Sie in welcher Reihenfolge?}

\subsection{Laderregler}
\subsubsection{Welche Voraussetzungen müssen Anlagen mit Akku ohne Laderegler haben? Wann wird ein Laderegler notwendig?}
\subsubsection{Geben Sie Verfahren zur Laderegelung in PV-Inselanlagen an und erläutern Sie deren Funktionsprinzip! Unter welchen Bedingungen ist welches Verfahren von Vorteil? (Beachten Sie auch den Kostenaspekt!)}
\subsubsection{Es soll eine defekte Batterie des Batteriesystems aus der Anlage gewechselt werden. Geben Sie die Reihenfolge Ihres Vorgehens schrittweise an.}

\subsection{Batteriesysteme}
\subsubsection{Welche Batterie-Typen werden in PV-Anlagen häufig eingesetzt? Nennen Sie Vor-und Nachteile! Was ist bei deren Laderegelung zu beachten?}
\subsubsection{Welche Anforderungen werden an einen Batterieraum gestellt?}
\subsubsection{Geben Sie die häufig eingesetzten Systemspannungen an! Was ist bei Gleichspannungsverbrauchern (insbesondere bei niedriger Spannung) im Vergleich zu Wechselstromverbrauchern zu beachten?}

\subsection{Wechselrichter in PV-Inselanlagen}
\subsubsection{Nach welchen Kriterien erfolgt die Auswahl eines Insel-Wechselrichters?}
\subsubsection{Welche Insel-Wechselrichter-Typen können zum Einsatz kommen (Kosten beachten)?}
\subsubsection{Woran erkennen Sie während des Betriebes einen „schlechten“ Wechselrichter? (niedriger Wirkungsgrad)?}
\subsubsection{Welche Anforderungen an den Einbauort des Wechselrichters müssen gewährleistet sein?}
\subsubsection{Unter welchen Umständen muss am WR einer PV-Inselanlage sofortige Lastabschaltung erfolgen?}
\subsubsection{Wie wirkt sich die benötigte Blindleistung auf die Dimensionierung des Wechselrichters aus?}
\subsubsection{Ein PC (einschließlich Peripherie) benötigt 120 W. Der Inselwechselrichter der unter 3. beschriebenen Anlage schaltet wegen Batterieerschöpfung nach 2 Tagen Betriebsdauer ab. Welche Möglichkeit des Dauerbetriebs der Anlage schlagen Sie vor. Begründen Sie die Realisierbarkeit.}

\subsection{Einstrahlung und Umgebungsbedingungen}
\subsubsection{Welche aus verschiedenen physikalischen Prinzipien resultierenden Messverfahren zur Erfassung der Globalstrahlung kennen Sie?}
\subsubsection{Welchen Einfluss haben diffuse und direkte Sonnenstrahlung auf die Leistung des PV-Generators?}
\subsubsection{Wovon hängt die Modultemperatur ab und welchen Einfluss hat sie?}