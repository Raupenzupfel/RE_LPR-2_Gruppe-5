\section{Vorbereitungsfragen}
\subsection{Photovoltaik-Inselanlage}
\subsubsection{Geben Sie vier Konzepte von PV-Inselanlagen mit je einem Einsatzbeispiel an}
\blindtext
\subsubsection{Die Anlagenkomponenten sollen im Inselsystem auf ihre Funktionsfähigkeit hin geprüft werden. Was messen Sie in welcher Reihenfolge?}
\blindtext

\subsection{Laderregler}
\subsubsection{Welche Voraussetzungen müssen Anlagen mit Akku ohne Laderegler haben? Wann wird ein Laderegler notwendig?}
\blindtext
\subsubsection{Geben Sie Verfahren zur Laderegelung in PV-Inselanlagen an und erläutern Sie deren Funktionsprinzip! Unter welchen Bedingungen ist welches Verfahren von Vorteil? (Beachten Sie auch den Kostenaspekt!)}
\blindtext
\subsubsection{Es soll eine defekte Batterie des Batteriesystems aus der Anlage gewechselt werden. Geben Sie die Reihenfolge Ihres Vorgehens schrittweise an.}
\blindtext

\subsection{Batteriesysteme}
\subsubsection{Welche Batterie-Typen werden in PV-Anlagen häufig eingesetzt? Nennen Sie Vor-und Nachteile! Was ist bei deren Laderegelung zu beachten?}
\blindtext
\subsubsection{Welche Anforderungen werden an einen Batterieraum gestellt?}
\blindtext
\subsubsection{Geben Sie die häufig eingesetzten Systemspannungen an! Was ist bei Gleichspannungsverbrauchern (insbesondere bei niedriger Spannung) im Vergleich zu Wechselstromverbrauchern zu beachten?}
\blindtext

\subsection{Wechselrichter in PV-Inselanlagen}
\subsubsection{Nach welchen Kriterien erfolgt die Auswahl eines Insel-Wechselrichters?}
\begin{itemize}
    \item Die Leistung des Umrichters muss an die Leistungs anforderungen des Insel-Netzes angepasst werden.
    \item Die Signalqualität der Spannung muss auch unter Last in den erforderten grenzen der Verbraucher bleiben.
    \item Der Wirkungsgradbereich muss nach dem Lastverhalten des Inselnetzes ausgelegt werden.
    \item Der Umrichter muss die gleiche Phasen-Anzahl besitzen, wie das System benötigt.
\end{itemize}
\subsubsection{Welche Insel-Wechselrichter-Typen können zum Einsatz kommen (Kosten beachten)?}
Mögliche Umrichter Typen:
\begin{itemize}
    \item mit/ ohne inneren Trafo
    \item 3 Level/ Multi Level
    \item 1/ 3 Phasig
\end{itemize} 
{\Huge nicht Final} % !!!
\subsubsection{Woran erkennen Sie während des Betriebes einen 'schlechten' Wechselrichter? (niedriger Wirkungsgrad)?}

\begin{itemize}
    \item Die Qualität der Spannung ist durch Oberschwingungen und Verzerrungen unzureichend.
    \item Der Umrichter produziert zu große mengen an Wärme im Nennlast betrieb.
    \item Der momentan Wirkungsgrad des Wechselrichters kann mit hilfe von \autoref{eq:230430_WR-Wirkungsgrad} bestimmt werden.
    \item Wenn der Wirkungsgrad bei Nennleistung deutlich unter einem wert von 90\% liegt, kann der Umrichter als 'schlecht' bezeichnet werden.
\end{itemize}

\begin{equation}
    \eta_{WR} = \frac{P_{WR,Ein}}{P_{WR,Aus}}
    \label{eq:230430_WR-Wirkungsgrad}
\end{equation}

\subsubsection{Welche Anforderungen an den Einbauort des Wechselrichters müssen gewährleistet sein?}
\begin{itemize}
    \item Der Einbauort muss vor Witterung schützen.
    \item Der Umrichter muss gut belüftet sein.
    \item Die Umgebungstemperatur des Umrichters, darf zu keiner zeit auhßerhalb seines angegebenen Temperaturbereichs liegen.
    \item Die Umrichter sollten für Wartungszwecke leicht zugänglich sein.
    \item Ein brand sollte möglichst schnell erkannt werden, durch z.B. Brandmelder und sollte keine anderen komponenten beeinflussen durch genügend Abstand oder abschirmung.
\end{itemize}
\subsubsection{Unter welchen Umständen muss am WR einer PV-Inselanlage sofortige Lastabschaltung erfolgen?}
Bei jedweder gefahr entdeckung für die verbraucher:
\begin{itemize}
    \item überspannungen durch z.B. Blitzeinschlag
    \item Kurzschlüsse durch z.B. Hochwasser
    \item Bei Feuer am Umrichter.
\end{itemize}
\subsubsection{Wie wirkt sich die benötigte Blindleistung auf die Dimensionierung des Wechselrichters aus?}
Durch zusätzlich benötigte Blindleistung muss der Umrichter nicht auf die Leistung des PV-Generators ausgelegt werden, sondern auf die scheinleistung, welches er ins netz geben muss.
Diese wird also zusätzlich innerhalb des Umrichters generiert, führt aber zu größeren Leistungflüssen.\cite{SMA_Q-Auslegung}
\subsubsection{Ein PC (einschließlich Peripherie) benötigt 120 W. Der Inselwechselrichter der unter 3. beschriebenen Anlage schaltet wegen Batterieerschöpfung nach 2 Tagen Betriebsdauer ab. Welche Möglichkeit des Dauerbetriebs der Anlage schlagen Sie vor. Begründen Sie die Realisierbarkeit.}
Es wird ein Umrichter benötigt, welcher sowohl Spannungsgeregelt als auch Stromgereglet betrieben werden kann, um zwischen Inselbetrieb und Netzbetrieb wechseln zu können.
\subsection{Einstrahlung und Umgebungsbedingungen}
\subsubsection{Welche aus verschiedenen physikalischen Prinzipien resultierenden Messverfahren zur Erfassung der Globalstrahlung kennen Sie?}
Es gibt 2 Physikalische Effekte, welche konventionell zu messung der Globalstrahlung verwendet werden. 
Der Photo-Effekt, welcher in Pyranomatern mit Halbleitersensoren verwendet wird. 
Dieser Basiert auf dem selben Prinzip wie eine Photovoltaikzelle und setzt die Einstrahlung in Proportion mit dem Strom.
Außerdem kann der Seebeck-Effejt verwendet werden, welcher in thermischen Pyranometern verwendet wird.
Hierbei wird eine von zwei miteinander verbundenen metallplatten durch die Bestrahlung aufgeheizt. 
Der Seebeck-Effekt besagt, dass durch eine Temperaturdifferenz zwischen 2 Leitermaterialien eine elektrische Spannung entsteht.\cite{Wiki-Seebeck}
Dies hat den Vorteil, dass ein viel größerer Teil des Spektrums gemessen wird.

\subsubsection{Welchen Einfluss haben diffuse und direkte Sonnenstrahlung auf die Leistung des PV-Generators?}
Der PV-Generaotr kan problemlos sowohl direkter als auch difuser Bestrahlung Leistung entnehmen. 
Jedoch hat die Difuse Einstrahlung eine geringere spezifische Leistung, da sie mehrweg durch die Atmosphäre zurückgeleht hat und somit einige Wellenlängen durch Absorptionsbänder herausgefiltert wurden.
\subsubsection{Wovon hängt die Modultemperatur ab und welchen Einfluss hat sie?}
Die Modultemperatue wird durch mehrere Fakotern beeinflusst.
\begin{itemize}
    \item Die Umgebungstemperatur
    \item Die Bestrahlungsstärke
    \item Die Einbauart (Hinterlüftet, aktive Kühlung etc.)
\end{itemize}
Hierbei ist zu beachten, dass nicht alle Wellenlängen des Sonnenlichts in einem Modul in Wärme umgewandelt werden, nur welche mehr Energie als die Bandlücke besitzen.
Diese erzeugen zwar auch ein Elektron-Loch-Paar, Generieren aber durch die übereregung Wärme innerhalb des Halbleiters.
Die Modul Temperatur hat direkten einfluss auf die Leistung des PV-Moduls.
Bei steigender Temperatur sinkt die Modulspannung deutlich und der Modulstrom steigt minimal.
