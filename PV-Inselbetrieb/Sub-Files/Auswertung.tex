\section{Auswertung}
\subsection{PV Wechselrichter}
\subsubsection{Wirkungsgrad Kennlinie}
	Laut Datenblatt hat der Wechselrichter Sunny Island 2012/2224 im Betrieb ohne Last einen Eigenverbrauch von 21W. In unserem Fall lag der Eigenverbauch ohne Last bei 29,7 W. Der Eigenverbrauch wird dabei nach Formel \ref{eq:2} aus der Leistung am LR im Leerlauf bestimmt.
%
\begin{equation}
	P_{ Eigenverbrauch }= U_{ LR} * I_{ LR }
\label{eq:2}
\end{equation}
%
\begin{table}[!ht]
    \centering
    \begin{tabular}{|l|l|l|l|l|l|l|l|l|l|l|l|l|l|l|l|}
    \hline
        Aufgabe & Last & T\_Gen & P\_Sonne & U\_Gen & I\_Gen & U\_LR & I\_LR & I\_Batt & I\_WR & P\_AC & U\_AC & I\_AC & f & Wirkungsgrad\_WR & cos(phi) \\ \hline
        WR\_allein & - & 28 & 690 & 39,3 & 0,84 & 25,4 & 1,17 & 0,15 & 0,97 & 12 & 229,83 & 0 & 50 & 48,71\% & 0 \\ \hline
        Staub\_1 & R-L & 26 & 550 & 33,6 & 7,84 & 24,5 & 8,9 & -10,5 & 21,9 & 510 & 228,5 & 3,04 & 55;99 & 95,05\% & 0,734193251 \\ \hline
        Staub\_2 & R-L & 24 & 460 & 32,2 & 12,49 & 23,3 & 15,6 & -27,2 & 43,4 & 942 & 227,47 & 4,33 & 50;75 & 93,15\% & 0,956398484 \\ \hline
        Staub\_3 & R-L & 23 & 471 & 36,8 & 7,66 & 23,2 & 10,94 & -46,2 & 57,3 & 1208 & 226,98 & 5,05 & 50;66,67 & 90,87\% & 1,053872239 \\ \hline
        S-Bahn\_Heizung & R & 23 & 601 & 34,8 & 16,15 & 23,2 & 20,51 & -33,7 & 55,7 & 1186 & 227,4 & 5 & 50;56 & 91,78\% & 1,043095866 \\ \hline
\label{tab:alles}
    \end{tabular}
\end{table}

Der Wirkungsgrad des Wechselrichters berechnet sich nach folgender Formel:
%
\begin{equation}
	\eta_{ WR} = \frac{ P_{AC} }{ P_{WR,Ein} }
\end{equation}
%
%
\begin{figure}[!h]
		\centering
		\includegraphics[width=0.5\textwidth]{Abbildungen/Kennlinie wr}
		\caption{Wirkungsgradkennlinie des Wechselrichters in Abhängigkeit der Auslastung}
		\label{fig:WRkennlinie}
\end{figure}
%
Dabei ist in Abbildung \ref{fig:WRkennlinie} zu erkennen, dass der Wirkungsgrad des Wechselrichters sich nicht linear verhält. Zunächst ist sehr niedrig und nähert sich dann einem Maximum an. Anschließend fällt der Wirkungsgrad wieder leicht ab, was an den steigenden Ohmschenverlusten liegt, die bei immer höher werdenen Strömen immer größer werden. Der maximale Wirkungsgrad liegt bei uns sogar bei 95 Prozent. Dies ist sogar 2 Prozent über dem Wirkungsgrad den der Hersteller angibt.
\subsubsection{Frequenz- und Spannungsstabilität}
Wie in unseren Messwerten zu sehen liegt die Spannung zwischen 226V und 230V. Die Spannung nimmt dabei mit steigender Leistung ab. Diese Werte passen auch sehr gut mit den Datenblattwert von 230 Volt zusammen. Die Frequenz bleibt dabei konstant bei 50 Hertz, wie man auf den Osziloskopbildern erkennen kann. Jedoch ist auch zu erkennen, dass es der Sinus nur im Leerlauf und beim anschließen der Straßenbahnheizung ohne Oberschwingungen daherkommt. Dies liegt daran, das hier ausschließlich wirkleistung benötigt wird. Beim Anschließen des STaubsaugers erkennt man die Oberschwingungen die entstehen. Dies liegt daran, dass der Staubsauer auch Blindleistung benötigt.
%
\begin{figure}[!h]
		\centering
		\includegraphics[width=0.5\textwidth]{Abbildungen/MergedImages}
		\caption{Von links nachrechts sieht man hier die Osziloskopbilder des Versuchs zuerst im Leerlauf, dann mit dem Staubsauger von STufe 1-3 und im letzten Bild wurde die STrßenbahnheizung angeschlossen.}
		\label{fig:oszi}
\end{figure}
%
\subsubsection{Leistungsbeiwert}
Der Leistungsbeiwert $ \phi $ berechnet sich wie folgt:
%
\begin{equation}
	cos(\phi)=\frac{ P_{ ACwirk } }{  I_{ AC }*U_{AC }}
\end{equation}
%
Dabei ist in der Tabelle zu sehen. Dass bei immer größerer Leistung beim Staubsauger Der Leistungsbeiwet sich immer näher an 1 annähert, was bedeutet, dass fast nur Wirkleistung benötigt wird.