% !TEX root = BZ.tex
%Einbindung Pakete
\documentclass[11pt,a4paper,oneside]{scrartcl}

%Pakete

%Style
\usepackage[left=2.5cm,right=2.5cm,top=2.0cm,bottom=2.5cm,footskip=1.5cm,head=25pt]{geometry}		
\usepackage{setspace}
\usepackage[bottom,hang]{footmisc}
\usepackage{longtable}

%Sprache
\usepackage[T1]{fontenc}  
\usepackage[utf8]{inputenc}
\usepackage[ngerman]{babel}
\usepackage{eurosym}
\usepackage{lmodern}
\usepackage{blindtext}

%Mathe- Umgebung
\usepackage{amsmath}
\usepackage{amssymb}
\usepackage{amsfonts}

%Grafiken
\usepackage{graphicx}																								
\usepackage{psfrag}
\usepackage{wrapfig}
\usepackage{caption}
\usepackage{subcaption}
\usepackage[headsepline,automark]{scrlayer-scrpage}

%Tabellenstil
\usepackage{tabularx}
\usepackage{multirow}
\usepackage{booktabs}
\usepackage{colortbl}

%Referenzen
\usepackage{url}
\usepackage{hyperref}
\bibliographystyle{unsrt}

%Das hyperref paket führt zu Problemen mit der Internet-Link Darstellung, da bei diesen mit aktivem Paket kein Zeilenumbruch möglich ist ggf. kann hier eine eigene Lösung gefunden werden
%\usepackage[hidelinks]{hyperref}

%Grafikpafad
\graphicspath{{./Abbildungen/}}

%Zusätzliche Pakete
\usepackage{caption}
\usepackage{units}
\usepackage{float}
\usepackage[]{xcolor}
\usepackage{colortbl}
\usepackage{hhline}

%Nachträglich hinzugefügte Pakete
%\usepackage{auto-pst-pdf}


%Initialisierung des Dokuments
\begin{document}

%Einbindung unterschiedlicher Befehlsänderungen
%  Referenzen Befehle
\newcommand{\fref}[1]{Abb. \ref{fig:#1}}
\newcommand{\tref}[1]{Tab. \ref{tab:#1}}
\newcommand{\mref}[1]{(\ref{eq:#1})}
\newcommand{\sref}[1]{Abschnitt \ref{section:#1}}

%Kopfzeile
\ohead[{\includegraphics[height=20pt]{HTWLogo}}]{\includegraphics[height=20pt]{HTWLogo}}
\renewcommand*\sectionmarkformat{}

%Farbe
\definecolor{HTWGreen}{cmyk}{0.55, 0.00, 1.00, 0.00}

%Nachträglich hinzugefügte Style Änderungen und Befehlsänderungen



\parindent 0cm

%Zeilenabstand	
\onehalfspacing			 									
\addtolength{\footskip}{-8mm}

%Titelseite
\begin{titlepage}

		\begin{figure}[h] 
				\begin{flushright}
			\includegraphics[width=0.3\textwidth]{HTWLogo}\\
				\end{flushright}
		\end{figure}
		
%Hochschulspezifische Informationen, Hochschule, Fachbereich, Adresse, etc.
	\begin{center}
		\vspace*{\fill}
		{\Large Hochschule f{\"u}r Technik und Wirtschaft Berlin}\\
			\bigskip
			Wilhelminenhofstra"se 75A, 12459 Berlin\\
			\bigskip
		Fachbereich 1 \\Ingenieurwissenschaften - Energie und Information\\Regenerative Energien (B)\\
		\vfill
%Versuchspezifische Angaben
		 \textcolor{HTWGreen}{\textbf{\Large{XXversuch vom xx.xx.xxxx}}}\\
		\textit{Betreuer*in: xxxxxxxxxxxxxxx}\\
		\textit{Gruppe: 5}\\
	\vfill
	\end{center}
\vfill
%Gruppenmitglieder Liste
\begin{table}[H]
			\centering
			\begin{tabular}{|l|c|}
			\hline
			\rowcolor[cmyk]{0.55, 0.00, 1.00, 0.00} \textbf{Name} & \textbf{Matrikelnummer}  \\
			\hline
			Johannes Tadeus Ranisch     & 578182\\
			\hline
			Markus Jablonka       & 580234\\
			\hline
			Niels Feuerherdt      & 577669\\
			\hline
			Katharina Jacob			 & 578522\\
			\hline
			Lukas Aust			 & 574051\\
			\hline
			\end{tabular}
			\end{table}
\end{titlepage}

\newpage

%Abstract (dieser ist für Laborprotokolle in der Regel nicht notwendig)
%\begin{abstract}

\begin{center}
\normalsize \textbf{Abstract}\\
\end{center}

\noindent
Kurze Zusammenfassung der Ergebnisse des Versuchs. Dieser wird in der Regel zuerst gelesen um zu identifizieren ob die vorliegende Arbeit Informationen enthält nach denen der Leser sucht.



\end{abstract}
% \newpage

\pagenumbering{roman}

%Inhaltsverzeichnis
\setcounter{tocdepth}{3}
\tableofcontents
\newpage

%Abbildungsverzeichnis, Tabellenverzeichnis und Abkürzungsverzeichnis
\listoffigures
\listoftables
%\addtocounter{table}{-1} 
\bigskip
\Large \textsf{\textbf{Abk"urzungsverzeichnis}}
\begin{longtable}{p{2 cm}p{8 cm}} 
PV & Photovoltaik\\
MPP & Maximum Power Point\\
\end{longtable}
\newpage

\pagenumbering{arabic}

%Inhaltlicher Teil
\section{Versuchsziele}
Der Versuch 'Photovoltaik-Inselbetrieb' beleuchtet zum tieferen Verständnis das Anlagenkonzept eines Photovoltaik-Inselsystems . Das System wird sowohl als System 
 für DC-Verbraucher als auch für AC-Verbraucher betrachtet. Das
 Verstehen und Aufstellen von Energiebilanzen wird geschult.\\
 Die Relevanz von Photovoltaik-Inselanlagen ist die vollständige Unabhängigkeit der Systeme. Die damit verbundenen Möglichkeiten reichen von der Versorgung kleinerer elektrischer Geräte wie Taschenrechner, Uhren oder GPS-Tracker bis hin zu der Möglichkeit eine Stromversorgung mit eigenem Netz für größere Geräte herzustellen.
 Mögliche Anwendungsbereiche sind hier die Versorgung von Geräten an Standorten
 an denen ein Netzanschluss nicht möglich ist, z.B. auf einer Bohr- oder Forschungsinsel
 oder dort wo noch kein Netz existiert und eine Inselanlage den ersten sicheren
 Stromversorger darstellt, wie es in einigen Entwicklungsländern bereits der Fall ist.\\
 %
\section{Theoretischer Hintergrund}
Der Versuch wurde anhand der Kenntnisse aus vorigen Veranstaltungen durchführt.\\
Die wichtigsten Erklärungen, Gleichungen und Gesetze werden in \autoref{sec:Vorbereitungsfragen}
bearbeitet.\\
\newpage
\section{Versuchsbeschreibung}
\label{section:Versuchsbeschreibung}
%
Der Prüfstand besteht aus einem Wasserkreislauf, angetrieben durch eine Pumpe wird Wasser durch ein Rohrsystem zur zu vermessenen Peltonturbine geleitet.\\
Im Verlauf des Rohrsystems werden sowohl der Druck als auch der Volumenstrom gemessen.
Hierfür werden Drucksensoren vom Typ \textit{PA3526} der Firma \textit{ifm electronic} genutzt, sowie eine Volumenstrommesseinheit.
Diese werden mit je einem Multimeter verschaltet von denen man dann einen Wert in \textit{mA} ablesen kann, welcher in den gesuchten Wert umgerechnet werden kann.\\
Eine Düse komprimiert dann den Wasserstrahl auf die Schaufeln der Peltonturbine.
Der Prüfstand im Stillstand ist ebenfalls in \autoref{fig:Aufbau_Stillstand} zu erkennen.\\
%
\begin{figure}[!h]
    \centering
    \includegraphics[scale=0.125]{Abbildungen/Aufbau Pelton.jpeg}
    \caption{Versuchsaufbau im Stillstand}
    \label{fig:Aufbau_Stillstand}
\end{figure}

An die Achse der Peltonturbine ist zusätzlich ein fremderregter Synchrongenerator mit einstellbarer Last gekoppelt.
An diesem werden die Drehzahl der Turbine mit einem Handmessgerät, sowie die mechanische Belastung am Generator mittels eines Kraftsensors gemessen.\\
Hier werden als Drehzahlmessgerät der \textit{VOLTCRAFT DT-10L} und als Kraftsensor der \textit{ME-Meßsysteme KD40S} verwendet.
Die Draufsicht auf die Kopplung und den Synchrongenerator ist in \autoref{fig:Synchrongenerator} dargestellt.\\

\begin{figure}[H]
    \centering
    \includegraphics[width=0.5\textwidth]{Abbildungen/Generator.jpeg}
    \caption{Synchrongenerator}
    \label{fig:Synchrongenerator}
\end{figure}

Um den Erregerstrom des Synchrongenerators einstellen und anzeigen lassen zu können ist der Generator in einer Sternschaltung an eine Schalttafel angeschlossen.
Des Weiteren kann an dieser Schalttafel auch der Lastwiderstand eingestellt werden.\\
Zusätzlich werden zwei Multimeter angeschlossen um den Phasenstrom, sowie die Leiterspannung messen zu können.\\
Der gesamte Aufbau der Schalttafel inklusive Multimeter ist in \autoref{fig:Schalttafel} zu sehen.

\begin{figure}[!ht]
    \centering
    \includegraphics[width=0.5\textwidth]{Abbildungen/Schalttafel.jpeg}
    \caption{Schalttafel}
    \label{fig:Schalttafel}
\end{figure}
\newpage
\section{Vorbereitungsfragen}
\label{sec:Vorbereitungsfragen}
\subsection{Wie ist die hydraulische Leistung definiert?}
%
\begin{equation}
	P_{ Eigenverbrauch }= U_{ LR} \cdot I_{ LR }\cdot \dot Q = \dot m \cdot g \cdot H
\label{eq:2}
\end{equation}
%
\subsection{Skizzieren Sie den typischen Verlauf einer Rohrleitungskennlinie}
Dies ist die typische Rohrleitungskennlinie.
%
\begin{figure}[!ht]
		\centering
		\includegraphics[width=0.5\textwidth]{Abbildungen/Rohrleitungskennlinie}
		\caption{Rohrleitungskennlinie bei vollständig geöfneter Düse}
		\label{fig:230512_Rohrleitungskennlinie}
\end{figure}
%
\subsection{Welche Proportionalität ergib sich bei Strömungsmaschinen zwischen Leistung und Drehzahl?}
\label{subsec:P_mech-n}
Die mechanische Leistung $P_{Mech.}$ ist in \autoref{eq:230512_MechanischeLeistung} definiert.
%
\begin{equation}
	P_{Mech.}= M \cdot 2 \cdot \pi \cdot n
\label{eq:230512_MechanischeLeistung}
\end{equation}
%
Dabei ist $n$ die Drezahl und $M$ das Moment. Somit ist die mechanische Leistung proportional zu der Drehzahl.
\subsection{Wie lässt sich der Betriebspunkt einer Pelton-Turbine einstellen?}
Der Betriebspunkt ist mit dem Volumenstrom/Strahldurchmesser,durch eine angelegte Last am Generator oder den Erregerstrom $I_{Err}$steuerbar. Dabei ist der optimale Betriebspunkt über die optimale Drehzahl zu finden. Dabei liegt die optimale Drehzahl bei der halben Austrittsgeschwindigkeit aus der Düse.

\subsection{Welcher hydraulische Parameter wird zur Regelung der Pelton-Turbine verändert?
Durch welche Einstellung passiert das?}
Die Düsennadel kann so eingestellt werden, dass der Durchflussquerschnitt sich verändert. Mit dem Durchflussquerschnitt lässt sich dann der Volumenstrom Q steuern und somit die Drehzahl der Perlton Turbine.
\section{Versuchsdurchführung}
\label{section:Versuchsdurchführung}
Der Versuch wurde am 23.06.2023 zwischen 11:30 Uhr und 13:00 Uhr durchgeführt. 
Die Wetterbedingungen am besagten Tag waren nahezu optimal für diesen Versuch. 
Dies zeichnete sich dadurch aus, dass der Himmel vollständig bedeckt war, was die 
Annahme zulässt, dass kaum direkte Bestrahlung Einfluss auf die Ergebnisse nehmen würde. 
Begonnen wurde der Versuch damit, dass der Raum von allem unnötigem Mobiliar befreit wurde.
Anschließend wurde der Raum vollständig vermessen. Fokus lag hierbei auf der Position 
der Lampen, sowie Fenster.\\
Das Ergebnis dieser Messungen ist der in \autoref{fig:grundriss} zu sehende Grundriss.
Aus dem entstandenen Grundriss ist zusätzlich ein einfaches dreidimensionales Modell erstellt wurden,
dieses ist in \autoref{fig:3D-Model} dargestellt.

\begin{figure}[H]
        \centering
        \begin{minipage}[b]{0.4\textwidth}
            \centering
            \includegraphics[width=\textwidth]{Abbildungen/Grundriss.jpg}
            \caption{Grundriss: Raum G319}
            \label{fig:grundriss}    
        \end{minipage}
        \hfill
        \begin{minipage}[b]{0.49\textwidth}
            \centering
            \includegraphics[scale=0.25]{Abbildungen/3D-Model.png}
            \caption{3D-Model: Raum G319}
            \label{fig:3D-Model}    
        \end{minipage}
\end{figure}

Sobald das Vermessen des Raumes abgeschlossen war, wurde die Versuchsgruppe
aufgeteilt. Ein Teil der Gruppe blieb im Raum und führte dort die Messungen durch
während der andere Teil die Messungen mit dem externen Luxmeter auf dem Dach durchführte. 
Um zu ermöglichen, dass die Messwerte an beiden Messstellen zeitgleich aufgenommen werden,
verständigte sich die Gruppe via Mobiltelefon.\\
Die Messwerte wurden sowohl von dem Laptop als auch von dem Luxmeter auf dem Dach
abfotografiert und für die anschließende Auswertung in einer Tabellenkalkulation zusammengetragen. 
Messwerte wurden an drei unterschiedlichen Positionen im Raum gemessen, hierbei wurde
das Messgerüst fortlaufend weiter von den Fenstern entfernt.
Für jede der drei Positionen wurden die folgenden drei Fälle betrachtet und mit jedem der 11 Sensoren ein Wert aufgenommen: 
\begin{itemize}
    \item Fenster geschlossen, Licht aus
    \item Fenster geöffnet, Licht aus
    \item Fenster geschlossen, Licht an
\end{itemize}


\newpage
\section{Auswertung}
\label{sec:Auswertung}


\subsection{Blockschaltbild}

\begin{figure}[H]
    \centering
    \includegraphics[width=\textwidth]{Abbildungen/Brennstoffzelle_Blockschaltbild.jpeg}
    \caption{Blockschaltbild}
    \label{fig:230628_Blockschaltbild}
\end{figure}


\subsection{Grafische Darstellung der Stackspannung, der Stackleistung, der Verbraucherleistung und des Wasserstoffverbrauchs}

\begin{figure}[H]
    \centering
    \includegraphics[width=0.8\textwidth]{Abbildungen/Aufgabe62_U und Vdot.png}
    \caption{Grafische Darstellung der Stackspannung und des Volumenstroms}
    \label{fig:230626_Stackspannung_Vdot}
\end{figure}

\begin{figure}[H]
  \centering
  \includegraphics[width=0.8\textwidth]{Abbildungen/Aufgabe62_P.png}
  \caption{Grafische Darstellung der Stackleistung}
  \label{fig:230626_Stackleistung}
\end{figure}


\subsection{Grafische Darstellung einiger Messgrößen im Bezug zum Verbracuherstrom}

\begin{figure}[H]
    \centering
    \includegraphics[width=0.8\textwidth]{Abbildungen/Aufgabe63_Leistungen_P.png}
    \caption{Grafische Darstellung der Leistungeneistung.}
    \label{fig:230626_Leistungen}
\end{figure}

\begin{figure}[H]
  \centering
  \includegraphics[width=0.8\textwidth]{Abbildungen/Aufgabe63_Leistungen_U.png}
  \caption{Grafische Darstellung der Stackspannung.}
  \label{fig:230626_Spannungen}
\end{figure}

\begin{figure}[H]
  \centering
  \includegraphics[width=0.8\textwidth]{Abbildungen/Aufgabe63_Leistungen_Ibat Vdot.png}
  \caption{Grafische Darstellung der Verbraucherleistung und des Wasserstoffverbrauchs.}
  \label{fig:230626_Ibat_Vdot}
\end{figure}

Um nun im nächsten Schritt den Wirkungsgrad des Stacks anhand des Heiz- und Brennwertes zu bestimmen, 
muss man den Wasserstoffverbrauch mit Hilfe des Heizwertes von $3 \frac{kWh}{m^3}$ und des Brennwerts von $3,54 \frac{kWh}{m^3}$ in eine Leistung umrechnen. 
Die tatsächlichen Leistungen an Stack und Gesamtsystem werden nun durch die vorher berechnete Leistung geteilt. 
Die folgende Gleichung zeigt diese Beispielhaft für den Stackwirkungsgrad:

\begin{equation}
 \eta= \frac{P_{Stack}}{H2_{Verbrauch_Stack}\cdot H2_{Brennwert}}
  \label{eq:230627_Beispiel_wirkungsgrad_Berechnung}
\end{equation}

$$\eta= \frac{1,506 kWh}{3,54 \frac{kWh}{m^3 }\cdot 0,954 \frac{m^3}{h}}=0,45$$

\begin{figure}[H]
    \centering
    \includegraphics[width=0.8\textwidth]{Abbildungen/Aufgabe63_Wirkungsgrade.png}
    \caption{Wirkungsgradverlauf für den Stack und das gesamte System.}
    \label{fig:230626_Wirkungsgrade}
\end{figure}

Dabei entstehen die Wirkungsgradbänder bei Brennstoffzellen aufgrund von Verlusten in den Reaktionskinetiken, 
ohmschen Verlusten, Massentransportverlusten und den Betriebsbedingungen. 
Diese Faktoren beeinflussen die Leistungsfähigkeit und den Wirkungsgrad des Systems.
Die Art der Brennstoffzelle und die verwendeten Materialien spielen eine Rolle bei der Ausprägung der Wirkungsgradbänder. 
Unterschiedliche Betriebsparameter können zu variierenden Wirkungsgradbändern führen. 
Es gibt verschiedene Arten von Brennstoffzellen mit jeweils spezifischen Charakteristika und Herausforderungen im Hinblick auf Wirkungsgradbänder.


\subsection{Diskussion der Kurven}

\autoref{fig:230626_Stackspannung_Vdot} und \autoref{fig:230626_Spannungen} zeigt die unterschiedlichen Systemspannungen in Abhängigkeit des Stackstroms und des Verbraucherstrom. 
Zu erkennen ist die charakteristische Brennstoffzellen Spannungskennlinie, die in \autoref{fig:230626_Spannungen} genauer beschrieben wird. 
Die Stack- als auch die Verbraucherleistung steigen in Abhängigkeit des zugeführten Wasserstoffes proportional an, was in \autoref{fig:230626_Stackleistung}, \ref{fig:230626_Leistungen} und \ref{fig:230626_Ibat_Vdot} veranschaulicht wurde. 
Außerdem ist zu erkennen, dass bei steigendem Strom, sowohl die Stack- als auch die Verbraucherspannung leicht, absinken. Die lässt sich auch an dem Abfallen des Wirkungsgrades ablesen. 
In \autoref{fig:230626_Wirkungsgrade} ist der Verlauf des Wirkungsgrads des Stacks (schwarz und blau) sowie des Gesamtwirkungsgrads (rot und magenta) dargestellt. 
Der Wirkungsgrad wurde jeweils unter Verwendung des oberen und unteren Heizwerts berechnet, und die Toleranzbänder sind farblich gekennzeichnet. 
Es ist zu beachten, dass der Wirkungsgrad des Stacks deutlich über 100 Prozent liegt, was eigentlich nicht möglich ist und im krassen Gegensatz zum Verlauf des Gesamtwirkungsgrads steht.
Der Grund dafür liegt darin, dass die Brennstoffzelle bei Inbetriebnahme die Membranen spült, um sie von Unreinheiten und Ablagerungen zu befreien. 
Dieses Spülen der Membranen erfolgt mit Wasserstoff, wodurch für einen bestimmten Zeitraum deutlich mehr Wasserstoff zur Verfügung steht. 
Dadurch beginnt die Zellreaktion, bevor eine Leistungsabnahme des Verbrauchers eintritt. Dies führt zu den beschriebenen Wirkungsgraden des Stacks, die während des Startvorgangs über 1 liegen.


\subsection{}




\subsection{Teillastverhalten}

Die Wirkungsgrade werden nach \autoref{eq:230627_Beispiel_wirkungsgrad_Berechnung} mit Hilfe der Nutzleistung, dem Wasserstoffverbrauch und dem Heiz- und Brennwert von Wasserstoff berechnet. 

\begin{table}[H]
    \caption{Teillastwirkungsgrade der Brennstoffzelle}
    \centering
        \begin{tabular}[pos]{|c|c|c|}
            \hline
            \rowcolor[HTML]{70AD47} 
            I=20A   & $\eta_{Gesamt,Brennwert}$               & $\eta_{Gesamt,Heizwert}$   \\\hline\hline
            1. Messung  & 38,4\%                            & 45,3\%                            \\ \hline
            2. Messung  & 39,5\%                            & 46,6\%                            \\\hline
        \end{tabular}
        \label{tab:20230628_Teillastwirkungsgrade}
\end{table}

Die Brennstoffzelle hat einen sehr guten Teillastwirkungsgrad. In der zweiten Teillast-Messung ist die Brennstoffzelle wärmer als zuvor und hat ihre optimale Betriebstemperatur erreicht. Die Bauteile der Schaltung (Kompressor und Kühlung) sind eingeschwungen. Es werden bessere Werte als in der ersten Messung bei 20A erreicht. Außerdem auffällig ist der hohe Batteriestrom, dieser ist so hoch, da die Batterie vorher (durch hohe Lastströme) entladen und nun wieder aufgeladen wird. 


\subsection{Energiebilanz für den Teillastfall}

\textbf{Messwerte noch ins Protokoll}
\newpage
%Quellenverzeichnis
\sloppy
\section{Quellen}
Laborversuchanleitung Tageslicht
\bibliography{Sub-Files/Libs/Tadeus_Quellen.bib}
\fussy
\newpage
\section{Anhang}
Die Messwerte sind in der Datei 'PV-Inselversuch-GR5.xls' zu finden.


\end{document}
