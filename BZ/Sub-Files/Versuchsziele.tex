\section{Versuchsziele}
\label{sec:Versuchsziele}
Der im Folgenden beschriebene Versuch "'Wasserstoff als Energieträger -
Automatisiertes Brennstoffzellensystem"' wurde am 16.06.2023 im Rahmen des Moduls
"'R68/LAB2 Labor Regenerative Energietechnik 2 (LPr)"'
auf dem "'Campus Wilhelminenhof"' der HTW Berlin
im Laborraum "'G311"' von 14:00 Uhr bis 15:30 Uhr durchgeführt.\\
Der Versuch wurde anhand der gegebenen Versuchsanleitung \cite{Anleitung}
vorbereitet und durchgeführt.\\
Die Zielstellung des Versuches ist es die in den vorausgegangenen Veranstaltungen
erlernten Kenntnisse zu den Grundlagen der Wasserstofftechnik zu festigen,
zu vertiefen und praktisch anzuwenden. Der Fokus soll hierbei besonders auf den
folgenden Punkten liegen:
\begin{itemize}
    \item Untersuchung und Verständnis des Betriebsverhaltens eines qualifizierten Brennstoffzellensystems
    \item Messtechnische Untersuchung der Brennstoffzellenparameter und deren grafische Auswertung
    \item Untersuchung verschiedener Betriebspunkte und -zustände zur Veranschaulichung des komplexen Systemverhaltens und der Bewertung der Einflussparameter
    \item Antizipation technischer Entscheidungshilfen und -kriterien
\end{itemize}
\vspace{\baselineskip}
Endgültig ist das Ziel die Brennstoffzelle so weit zu verstehen, dass diese
in unterschiedlichsten Anwendungsmöglichkeiten genutzt werden kann, hierzu
gehören die Versorgung von mit Wasserstoff betriebenen Fahrzeuge oder die
Kraft-Wärme-Kopplung in BHKW's.