\section{Auswertung}
\label{sec:Auswertung}


\subsection{Blockschaltbild}
\subsection{Grafische Darstellung der Stackspannung, der
Stackleistung, der Verbraucherleistung und des Wasserstoffverbrauchs}
\begin{figure}[H]
    \centering
    \includegraphics[width=0.8\textwidth]{plot1}
    \caption{Grafische Darstellung der Stackspannung, der
    Stackleistung, der Verbraucherleistung und des Wasserstoffverbrauchs.}
    \label{fig:plot1_26062023}
  \end{figure}
\subsection{Grafische Darstellung einiger Messgrößen im Bezug zum Verbracuherstrom}
\begin{figure}[H]
    \centering
    \includegraphics[width=0.8\textwidth]{grafik2}
    \caption{Grafische Darstellung der Stackspannung, der
    Stackleistung, der Verbraucherleistung und des Wasserstoffverbrauchs.}
    \label{fig:plot2_26062023}
  \end{figure}
  Um nun im nächsten Schritt den Wirkungsgrad des Stacks anhand des Heiz- und Brennwertes zu bestimmen muss man den Wasserstoffverbrauch mit Hilfe des Heizwertes von $3 \frac{kWh}{m^3}$ und des Brennwerts von $3,54 \frac{kWh}{m^3}$
von Wasserstoff kann unser Wasserstoff Verbrauch nun in eine Leistung umgerechnet werden. Die tatsächlichen Leistungen an Stack und Gesamtsystem werden nun durch die vorher berechnete Leistung geteilt. Die folgende Gleichung zeigt diese Beispielhaft für den Stackwirkungsgrad:
\begin{equation}
 \eta= \frac{P_{Stack}}{H2_{Verbrauch_Stack}*H2_{Brennwert}}= \frac{1,506 kWh}{3,54 \frac{kWh}{m^3 }\cdot 0,954 \frac{m^3}{h}}=0,45
  \label{eq:230627_Beispiel_wirkungsgrad_Berechnung}
\end{equation}
\begin{figure}[H]
    \centering
    \includegraphics[width=0.8\textwidth]{plot3}
    \caption{Wirkungsgradverlauf für den Stack und das gesamte System.}
    \label{fig:plot3_26062023}
  \end{figure}
\subsection{Diskussion der Kurven}
\autoref{fig:plot1_26062023} und \autoref{fig:plot2_26062023} zeigt die unterschiedlichen Systemspannungen in Abhängigkeit des
Verbraucherstroms. Zu erkennen ist die charakteristische Brennstoffzellen Spannungskennlinie (hier
blau), die in \autoref{fig:plot2_26062023} genauer beschrieben wird. Die Stack- als auch die Verbraucherleistung
steigen in Abhängigkeit des zugeführten Wasserstoffes proportional an. In den \autoref{fig:plot1_26062023} und \autoref{fig:plot2_26062023} ist weiterhin zu erkennen, dass bei einem Verbraucherstrom ab ca. 40 A, sowohl die Stack- als
auch die Verbraucherspannung leicht, absinken. Die lässt sich auch an dem Abfallen des
Wirkungsgrades ablesen. \textbf{Es fehlt noch: warum entstehen Wirkungsgradbänder}
\subsection{}
\subsection{Teillastverhalten}

Leistung prinzipiell mit höherer Temperatur größer, außerdem Batteriestrom höher? 
Steigt der Teillastwirkungsgrad mit steigender Temperatur (nicht über betriebstemperatur) oder ist das unsinnig?
Warum ist der Batteriestrom auf einmal so hoch? Puffert die Batterie die Leistung aus die nicht mehr von der BZ bereitgestellt werden kann durch die höhere temperatur?



\subsection{Energiebilanz für den Teillastfall}