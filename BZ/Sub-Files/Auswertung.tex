\section{Auswertung}
\label{sec:Auswertung}


\subsection{Blockschaltbild}
\subsection{}
\subsection{}
\subsection{Diskussion der Kurven}
\subsection{}
\subsection{Teillastverhalten}

Die Brennstoffzelle hat einen sehr guten Teillastwirkungsgrad. In der zweiten Teillast-Messung ist die Brennstoffzelle wärmer als zuvor und hat ihre optimale Betriebstemperatur erreicht. Die Bauteile der Schaltung (Kompressor und Kühlung) sind eingeschwungen. Es werden bessere Werte als in der ersten Messung bei 20A erreicht. Außerdem auffällig ist der hohe Batteriestrom, dieser ist so hoch, da die Batterie vorher (durch hohe Lastströme) entladen und nun wieder aufgeladen wird. 



\subsection{Energiebilanz für den Teillastfall}