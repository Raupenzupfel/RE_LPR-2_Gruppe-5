\section{Versuchsdurchführung}
\label{section:Versuchsdurchführung}

Begonnen wird der Versuch durch eine Einführung in den Prüfstand durch die Versuchsleitung, diese öffnet ebenfalls die Wasserstoffzufuhr und startet das System.\\
Anschließend sind der Umgebungsdruck und die Umgebungstemperatur, sowie der Druck am Reduzierventil aufzunehmen.\\
Dann beginnt die erste Messreihe, bei dieser werden für jeden Betriebspunkt drei Werte im Abstand von einer Minute aufgenommen.
Nach dem Einstellen eines neuen Zustandes wird ebenfalls eine Minute gewartet.
Es werden die Startwerte aufgenommen und dann die Last auf die vorgegebenen Lastpunkte
eingestellt. Diese sind zuerst 0 A und 1 A und dann Werte im Intervall von 5A beginnend bei 5A und bis 40 A. Der letzte Wert ist der höchstmögliche Wert, bevor das System notabschaltet.
Dieser Punkt ist hier 43 A.\\
Der Messpunkt der zweiten Messreihe wird direkt nach der letzten Messung von Messreihe 1 durchgeführt, hierbei wird die Last schlagartig wieder auf 20 A gestellt.\\
Für jeden Messpunkt, beider Messreihen, werden Verbraucherspannung, Stacktemperatur, -strom, -spannung und -leistung aufgenommen, sowie die Wasserstoffzufuhr, Batteriespannung und -strom und der Verbrauch des Kompressors und der Kühlung aufgenommen.\\
Alle Messwerte sind in der beigefügten Tabellenkalkulation notiert.\\