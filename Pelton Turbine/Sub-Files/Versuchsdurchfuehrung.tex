\section{Versuchsdurchführung}
Der Versuch wurde am in \autoref{section:Versuchsbeschreibung} beschriebenden Prüfstand entsprechend der Anweisungen aus der Versuchsanleitung \cite[S.11-14]{Anleitung} durchgeführt.\\
Zu Beginn des Versuches wird die Pumpe bei geschlossenem Kugelhahn gestartet und dieser anschließend geöffnet um dann die Druckmessleitung im Wasserbecken zu entlüften.\\
Anschließend beginnt die erste Messreihe bei welcher bei unterschiedlich weit geöffneter Düse der Druck und der Volumenstrom im System gemessen wird.
Die Schrittweite der Messungen beträgt hier 0,25 Umdrehungen und geht von 0 bis 1,5 Umdrehungen, die Umdrehungen beziehen sich hierbei auf den drehbaren Hahn zur Einstellung der Düsenöffnung.\\
Die Messergebnisse wurden hierbei in der beigefügten Tabellenkalkulation Pelton-Turbine.xlsx im Tabellenblatt "5.1" dokumentiert.\\
Ziel dieser Messreihe ist die grafische Darstellung und grundlegende Ermittlung der Pumpenkennlinie.\\
\newline
Das Ziel der zweiten Messreihe ist die Aufnahme aller benötigten Werte um im Anschluss den Wirkungsgrad und den optimalen Betriebspunkt der Pelton-Turbine,
 sowie den Verlustbeiwert der Düse zu ermitteln.\\
 Hierfür wird der Lastwiderstand des Synchrongenerators auf 5,0 k$\Omega$ eingestellt und der Erregerstrom beginnend von 0 mA in Schritten von 30 mA bis auf 300 mA erhöht.\\
 Sobald der Erregerstrom den Wert von 300 mA erreicht hat wird dann der Lastwiderstand in unregelmäßigen Schritten entsprechend der Vorgaben aus der Versuchsanleitung \cite[S.12]{Anleitung} gesenkt.\\
 Während dieses Ablaufes wird am Laborcomputer die Kraft am Hebelarm des Generators gemessen.\\
 Mit einem Handmessgerät wird die Drehzahl der Achse der Turbine gemessen.\\
 Und mittels der vier Multimeter werden die Mesströme für Druck und Volumenstrom, sowie die Leiterspannung und der Phasenstrom gemessen.\\
 Die Ergebnisse dieser Messungen wurden in im Tabellenblatt "5.2" der bereits genannten und beigefügten Tabellenkalkulation notiert.\\
