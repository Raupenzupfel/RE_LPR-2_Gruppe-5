\section{Versuchsdurchführung}
Der Versuch wird im, in \autoref{section:Versuchsbeschreibung} beschriebenen, Prüfstand entsprechend der Anweisungen aus der Versuchsanleitung \cite[S.11-14]{Anleitung} durchgeführt.\\
Zu Beginn des Versuches wird die Pumpe bei geschlossenem Kugelhahn gestartet und anschließend geöffnet um dann die Druckmessleitung im Wasserbecken zu entlüften.\\
Anschließend beginnt die erste Messreihe, bei welcher bei unterschiedlich weit geöffneter Düse, der Druck und der Volumenstrom im System gemessen wird.
Die Schrittweite der Messungen beträgt hier 0,25 Umdrehungen und geht von 0 bis 1,5 Umdrehungen. Die Umdrehungen beziehen sich hierbei auf den drehbaren Hahn zur Einstellung der Düsenöffnung.\\
Die Messergebnisse werden in der beigefügten Tabellenkalkulation "'Pelton-Turbine.xlsx"' im Tabellenblatt "'5.1"' dokumentiert.\\
Ziel dieser Messreihe ist die grafische Darstellung und grundlegende Ermittlung der Pumpenkennlinie.\\
\newline
Das Ziel der zweiten Messreihe ist die Aufnahme aller benötigten Werte für die Ermittlung des Wirkungsgrades und des optimalen Betriebspunkts der Pelton-Turbine,
 sowie dem Verlustbeiwert der Düse.\\
 Hierfür wird der Lastwiderstand des Synchrongenerators auf 5,0 k$\Omega$ eingestellt und der Erregerstrom beginnend von 0 $mA$ in Schritten von 30 $mA$ bis auf 300 $mA$ erhöht.\\
 Sobald der Erregerstrom den Wert von 300 $mA$ erreicht hat wird der Lastwiderstand in unregelmäßigen Schritten entsprechend der Vorgaben aus der Versuchsanleitung \cite[S.12]{Anleitung} gesenkt.\\
 Während dieses Ablaufes wird am Laborcomputer die Kraft am Hebelarm des Generators gemessen.\\
 Mit einem Handmessgerät wird die Drehzahl der Achse der Turbine gemessen.\\
 Mittels der vier Multimeter werden die Mesströme für Druck und Volumenstrom, sowie die Leiterspannung und der Phasenstrom gemessen.\\
 Die Ergebnisse dieser Messungen werden in Tabellenblatt "'5.2"' der bereits angeführten und beigefügten Tabellenkalkulation notiert.\\
