\section{Auswertung}
In diesem Abschnitt erfolgt die Auswertung der aufgenommenen Messwerte des Versuchs, die genaue Aufgabenstellung hängt vom Versuchsthema ab. Ergebnisse sind graphisch, ggf. zusätzlich auch tabellarisch darzustellen.\\

Diagramme der Auswertung erhalten ebenfalls eine Bildunterschrift mit fortlaufender Nummerierung, Tabellen eine Bildüberschrift (siehe z.B. Tabelle 1). Jedes Ergebnis ist im Anschluss mit ein bis zwei Sätzen zu kommentieren damit auch Leser, die nicht mit dem Stoff vertraut sind, die Auswertung nachvollziehen können.\\

Für die \textbf{Fehlerbetrachtung} ist zu prüfen, welche Toleranzen sich aufgrund der begrenzten Genauigkeit der verwendeten Messgeräte ergeben.\\

Eine vollständige, tabellarische Auflistung der Messwerte einschließlich möglicher Anmerkungen (\glqq Noch zu wenig Wind, WEA dreht nicht\grqq{}) muss im \textbf{Anhang} erfolgen. Ggf. erfordert die Aufgabenstellung auch die tabellarische Darstellung einzelner Werte im Rahmen der Auswertung.\\

\begin{table}
\centering
\caption{Speicherkapazitäten für H\textsubscript{2} und Methan in Deutschland (Daten: [UBA10; LBG13], Stand 2013)}
\arrayrulecolor{black}
\begin{tabular}{|l|l|l|l|} 
\hhline{~---|}
\multicolumn{1}{l|}{\multirow{2}{*}{}}                                       & {\cellcolor[rgb]{0.463,0.725,0}}                                                                                                                                            & \multicolumn{2}{l|}{{\cellcolor[rgb]{0.463,0.725,0}}\begin{tabular}[c]{@{}>{\cellcolor[rgb]{0.463,0.725,0}}l@{}}\textbf{Speicherkapazität~}\\\textbf{in TWh für}\\ \end{tabular}}  \\ 
\hhline{~>{\arrayrulecolor[rgb]{0.463,0.725,0}}->{\arrayrulecolor{black}}--|}
\multicolumn{1}{l|}{}                                                        & \multirow{-2}{*}{{\cellcolor[rgb]{0.463,0.725,0}}\begin{tabular}[c]{@{}>{\cellcolor[rgb]{0.463,0.725,0}}l@{}}\textbf{Arbeitsgasvolumen~}\\\textbf{in Mio. m\textsuperscript{3}}\end{tabular}} & {\cellcolor[rgb]{0.463,0.725,0}}\textbf{H2-Gas~} & {\cellcolor[rgb]{0.463,0.725,0}}\textbf{Methan}                                                                                 \\ 
\hline
Porenspeicher in Betrieb                                                     & 10,6                                                                                                                                                                        & ---                                              & 106                                                                                                                             \\ 
\hline
\rowcolor[rgb]{0.827,0.886,0.729} Kavernenspeicher in Betrieb                & 12,1                                                                                                                                                                        & 36                                               & 121                                                                                                                             \\ 
\hline
Kavernenspeicher in Betrieb, Bau oder Planung                                & 10,9                                                                                                                                                                        & 33                                               & 109                                                                                                                             \\ 
\hline
\rowcolor[rgb]{0.827,0.886,0.729} Kavernenspeicher Langfrist-Gesamtpotenzial & 36,8                                                                                                                                                                        & 110                                              & 368                                                                                                                             \\
\hline
\end{tabular}
\end{table}


