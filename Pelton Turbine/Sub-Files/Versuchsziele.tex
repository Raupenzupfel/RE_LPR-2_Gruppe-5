\section{Versuchsziele}

Ein (Labor-)Experiment ist eine methodisch angelegte Untersuchung zur empirischen Gewinnung von Information. Protokolle sind notwendig, um eine Nachprüfung der eigenen Unter-suchungen durch andere zur ermöglichen. In vielen Fällen sollte vor der Durchführung eine Vermutung (Hypothese) darüber, wie das Experiment verlaufen wird, formuliert worden sein.\\

In der Entwicklung von Produkten werden zunehmend Modelle (modellasierte Entwicklung) eingesetzt, lange bevor Prototypen zur Untersuchung vorliegen. Eine Überprüfung (Validie-rung) der erstellten Modelle ist daher in der Praxis ein immer wichtigerer Grund für Vermes-sungen, die dann durch Protokolle/Berichte dokumentiert werden.\\

Erklären Sie kurz
%
\begin{enumerate}
\item das Ziel des Versuchs
\item praktische Einsatzgebiete der im Labor betrachteten Technologie
\item ggf. die die Hypothese die überprüft werden soll
\item ggf. was für ein Modell validiert werden soll.
\item Welche in der Praxis relevanten Informationen ggf. aus dem Versuch gewonnen werden können?
\item Ggf. Richtlinien oder Vorschriften, auf die Bezug genommen wird.
\end{enumerate}
%