\section{Theoretischer Hintergrund}

\begin{equation}
    P_{Hyd.} = \rho \cdot g \cdot Q \cdot H
    \label{eq:230514_hydraulische_Leistung}
  \end{equation}

  \begin{equation}
    \eta_{Turbine} = \frac{P_{Hyd.}}{P_{Mech.}}
    \label{eq:230512_Turbinenwirkungsgrad}
  \end{equation}

  \begin{equation}
    u_{opt} = \frac{c_{0}}{2}=\pi \cdot u_{opt} \cdot d_{2}
    \label{eq:230515_umlaufgeschwindigkeit}
  \end{equation}

  \begin{equation}
    c_{0}=\frac{4 \cdot Q}{\pi \cdot D_{D}^2}
      \label{eq:230515_austrittsgeschwindigkeit}
    \end{equation}
    
    
    \begin{equation}
     n_{opt}=\frac{c_{0}}{2 \cdot \pi \cdot d_{2}}=\frac{2 \cdot Q}{d_{2} \cdot \pi^2  \cdot D_{D}^2}=1766,23  min^-1
    \label{eq:230515_n-optimal}
    \end{equation}

    \begin{equation}
        \delta_c = c_{0.H} - c_{0.Q} = ( 2\cdot g \cdot H_T)^{0,5} - 4\cdot \frac{Q}{\pi \cdot D_D^2}
    \end{equation}

    \begin{equation}
        H_{V.D} = \frac{\Delta c^2}{2\cdot g}
    \end{equation}
        
     \begin{equation}
        \zeta_D = \frac{H_{V.D} \cdot 2 \cdot g}{c_{o.Q}^2}
    \end{equation}