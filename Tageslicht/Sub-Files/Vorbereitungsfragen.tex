\section{Vorbereitungsfragen}
\label{sec:Vorbereitungsfragen}

Der k1 wird mit der \autoref{eq:230624_k1} berechnen, wobei $A'_F$ die Lichtdurchlässige fkäche und $A_F$ die Fläche der Rohbauöffnung darstellt.
Die Rahmen fläche lässt sich mittels der Rahmendicke und dem gegebenen Umfang der Fenster berechenen: $A_{Rahmen} = U_{Fenster} \cdot d_R$. 
Daraus widerum lässt sich die Lichtdurchlässige Fläche berechnen:  $A'_F = A_{Fenster} - A_{Rahmen}$.
Es wurde für 'Good Practice' von einer Rhamendicke von $d_{R,GP} = 5cm$ und für den 'Worstcase' $d_{R,WC} = 7 cm$ ausgegangen.
Daraus ergeben sich $k_{1,GP} = 0,8542$ und $k_{1,WC} = 0,7958$ welche auch in \autoref{tab:230426_Korrektur-Faktoren} zusammengefasst wurden.
Der $k_2$ kann mittels einer Tabelle in der DIN 5034-3 abgeschätzt werden.
Der Bestcase der verschmutzung wäre $k_2 = 0,9$ der Worstcase hingegen baträgt $k_2 = 0,5$.
Da 'Good Practice' allerdings nicht der Bestcase ist, wird von $k_{2,GP} = 0,8$ ausgegangen.
Der Korrektur Faktor für nicht senkrechten Lichteinfall hingegen hat einen Pauschalwert von $k_3 = 0,85$ und der Transmissionsgrad wurde als $T = 0,9$ vorgegeben.
Nun kann der Effektive Transmissionsgrad mit der \autoref{eq:230624_Lichttransmissionsgrad} berechnet werden und wurde, so wie alle anderen Faktoren, in \autoref{tab:230426_Korrektur-Faktoren} zusammengefasst.


\begin{equation}
    k_1 = \frac{A'_F}{A_F}
    \label{eq:230624_k1}
\end{equation}

\begin{equation}
    \tau = k_1 \cdot k_2 \cdot k_3 \cdot T
    \label{eq:230624_Lichttransmissionsgrad}
\end{equation}


\begin{table}[H]
    \caption{Zusammenfassung der Korrektur Faktoren}
    \centering
    \begin{tabular}{|c|c|c|}
    \hline
    \rowcolor[HTML]{70AD47} 
    Korrektur Faktor                                & Good Practice & Worst Case \\\hline
    \rowcolor[HTML]{CFE5A8} 
    $k_1$ - Rahmen und Sprossen                     & 85,42\%       & 79,58\%    \\\hline
    \rowcolor[HTML]{A9D08E} 
    $k_2$ - Verschmutzung                           & 80\%          & 50\%       \\\hline
    \rowcolor[HTML]{CFE5A8} 
    $k_3$ - nicht senkrechter Lichteinfall          & 85\%          & 85\%       \\\hline
    \rowcolor[HTML]{A9D08E} 
    T - Transmissionsgrad                           & 90\%          & 90\%       \\\hline
    \rowcolor[HTML]{CFE5A8} 
    $\tau$ - Effektiver Lichttransmissionsgrad      & 52,28\%       & 30,44\%   \\\hline
    \end{tabular}
    \label{tab:230426_Korrektur-Faktoren}
\end{table}

\subsection{Was unterscheidet Ihre Ergebnisse vom Verlauf des Tageslichtquotienten D in Abb. 5?}


\begin{figure}[H]
    \centering
    \includegraphics[width=0.7\textwidth]{Abbildungen/abb5.png}
    \caption{Abbildung 5 aus der Versuchsanleitung }
    \label{fig:abb5}
\end{figure}

Die Ergebnisse und die Abbildung unterscheiden sich insofern, dass in den Berechnungen nicht der Tageslichtquotient D, sondern der Transmissionsgrad und die Verminderungs- bzw. Korrekturfaktoren bestimmt wurden.
Dazu kommt, dass die Abbildung die Tiefe des Raumes berücksichtigt während die Berechnungen nur einen einzelnen Wert ergeben. 
Die Berechnungen berücksichtigen dafür durch die zwei unterschiedlichen Szenarien mögliche Schwankungen durch beispielsweise Verschmutzungen.

\subsection{Wie würden Sie die Steuerung der Beleuchtung anpassen?}




\subsection{Von welchen baulichen Bedingungen hängt der Himmelslichtanteil DH ab?}

Der Himmelslichtanteil ist von vielen Faktoren abhängig. Dazu zählen:
\begin{itemize}
\item Fenstergröße und Position beeinflussen das einfallende Licht stark. Größer, weiter oben liegende Fenster lassen mehr Lichteinfall zu als kleinere. 
\item Die Umgebung rund um Fenster bzw. Gebäude, wie Bäume, Hügel, andere Gebäude usw. können Abschattungen oder Ähnliches verursachen.
\item Innenausstattung wie Vorhänge, Rollläden, Farbe der Wände etc. 
\item Wetterbedingungen
\item Die Transmissionsgrade der Fenster und deren Verschmutzung

\end{itemize}

\subsection{Welche baulichen Maßnahmen könnten ihn verbessern?}
Der Himmelslichtanteil kann durch größere Fenster, den Einbau von Oberlichtern oder Lichtschächten und reflektierende Materialien verbessert werden. Außerdem kann das Fällen von Bäumen vor dem Fenster oder sogar im Extremfall der Abriss eines in der Umgebung befindlichen Gebäudes den Lichteinfall verbessern.

\subsection{Um wie viel reduziert sich Ea, wenn bei gleichmäßig bedecktem Himmel der Horizont
gleichförmig 10° über der Horizontalen liegt?}

Allgemein gesprochen führt eine Erhöhung des Horizonts über der Horizontalen zu einer Verringerung des einfallenden Lichts.
Da bei vollständig bedecktem Himmel davon ausgegangen werden kann, dass der Lichteinfall von überall gleich stark ist muss die wegfallende Fläche durch die Erhöhung des Horizontes berechnet werden. Daraus ergibt sich bei einer Halbkugel von $180^{\circ}-20^{\circ}=160^{\circ}$ was 89 \% des ursprünglichen Wertes entspricht. $E_a$ reduziert sich also um ca. 11 \%. 




