\section{Auswertung}
\label{sec:Auswertung}
\subsection{Bestrahlungstärke pro Fall}
\begin{figure}[H]
    \centering
    \includegraphics[width=0.85\textwidth]{plot1}
    \caption{Bestrahlungsstärke für Fall geschlossenes Fenster, ohne Kunstlicht}
    \label{fig:plot1_28062023}
  \end{figure}
 
  \begin{figure}[H]
    \centering
    \includegraphics[width=0.85\textwidth]{plot2}
    \caption{Bestrahlungsstärke für Fall offenes Fenster, ohne Kunstlicht}
    \label{fig:plot2_28062023}
  \end{figure}

  \begin{figure}[H]
    \centering
    \includegraphics[width=0.85\textwidth]{plot3}
    \caption{Bestrahlungsstärke für Fall geschlossenes Fenster, mit Kunstlicht}
    \label{fig:plot3_28062023}
  \end{figure}

  Die Schlussfolgerungen aus der Auswertung der vorliegenden Daten sind wie folgt: Die äußeren Messwerte wurden gemittelt, um eine durchschnittliche Bestrahlungsstärke zu erhalten, während der mittlere Wert keinen Durchschnitt darstellt, sondern einen spezifischen Messwert repräsentiert.
Der Sollwert der Beleuchtungsstärke für Arbeitsplätze ist auf 500 Lux festgelegt. Messreihe 1 bei geschlossenem Fenster und Messreihe 2 bei offenem Fenser und ausgeschaltetem Licht erfüllen die Anforderung für Arbeitsplätze nicht. Einzig bei eingeschalteter zusätzlicher Beleuchtung werden die 500 Lux erreicht.
Außerdem lässt sich feststellen, dass die Beleuchtungsstärke im betrachteten Raum von Fenster zu Tür hin abnimmt. Zu erkennen ist auch, dass das Primsmenglas die Beleuchtungsstärke reduziert. Diese Abnahme zeigt sich in allen drei betrachteten Szenarien.
Des Weiteren zeigt sich, dass das Öffnen des Fensters, insbesondere wenn Prismenglas verwendet wird, zu einer signifikanten Steigerung der Beleuchtungsstärke führt. In der rechten Raumhälfte ist eine deutliche Zunahme zu beobachten, während in der linken Raumhälfte eine moderate Steigerung festzustellen ist.
Die Verwendung von künstlichem Licht trägt dazu bei, die Beleuchtungsverhältnisse im Raum zu stabilisieren. Allerdings führt dies auch zu erhöhten Beleuchtungsverhältnissen in der Nähe des Fensters. Um den Energieverbrauch zu senken und eine ausgewogenere Beleuchtungsstärke zwischen künstlichem Licht und Tageslicht zu erzielen, könnte eine Dimmung der Leuchten im Fensterbereich in Betracht gezogen werden.
Abschließend ist anzumerken, dass Tageslichtquotienten üblicherweise in Arbeitsbereichen bestimmt werden. Diese Quoten sollten idealerweise zwischen 1\% und 10\% liegen, um eine angemessene Beleuchtung für die durchzuführenden Arbeiten zu gewährleisten.

\newpage
\subsection{Tageslichtkoeffizient Messung 1}
\begin{table}[H]
  \centering
  \caption{Messreihe 1, äußere Bestrahlungsstärke, Tageslichtkoeffizient}
  \label{tab:170723_Messung1}
  \small
  \resizebox{\columnwidth}{!}{%
  \begin{tabular}{|l|l|l|l|l|l|l|l|l|l|l|l|l|l|}
  \hline
  \rowcolor[HTML]{70AD47} 
  {\color[HTML]{343434} \textbf{Abstand Fenster}} & {\color[HTML]{343434} \textbf{Sensor Position Raumtiefe}} & {\color[HTML]{343434} \textbf{1}} & {\color[HTML]{343434} \textbf{2}} & {\color[HTML]{343434} \textbf{3}} & {\color[HTML]{343434} \textbf{4}} & {\color[HTML]{343434} \textbf{5}} & {\color[HTML]{000000} \textbf{$E_A$ in Lux}} & {\color[HTML]{343434} \textbf{Sensor Position Raumtiefe2}} & {\color[HTML]{343434} \textbf{Position1}} & {\color[HTML]{343434} \textbf{Position 2}} & {\color[HTML]{343434} \textbf{Position 3}} & {\color[HTML]{343434} \textbf{Position 4}} & {\color[HTML]{343434} \textbf{Position 5}} \\ \hline
  \rowcolor[HTML]{C6E0B4} 
  300cm                                             & lx   Messreihe Innen 1 in Lux                                    & 588                               & 552                               & 156                               & 0                                 & 24                                & \cellcolor[HTML]{A9D08E}10980         & D   Messreihe Innen 1 in \%                                     & 5,36                                      & 5,03                                       & 1,42                                       & 0,00                                       & 0,22                                       \\ \hline
  \rowcolor[HTML]{E2EFDA} 
  400cm                                             & lx   Messreihe Innen 2 in Lux                                   & 300                               & 264                               & 156                               & 96                                & 60                                & \cellcolor[HTML]{A9D08E}10980         & D   Messreihe Innen 2 in \%                                    & 2,73                                      & 2,40                                       & 1,42                                       & 0,87                                       & 0,55                                       \\ \hline
  \rowcolor[HTML]{C6E0B4} 
  500cm                                             & lx   Messreihe Innen 3 in Lux                                   & 36                                & 132                               & 60                                & 12                                & 12                                & \cellcolor[HTML]{A9D08E}11950         & D   Messreihe Innen 3 in \%                                     & 0,30                                      & 1,10                                       & 0,50                                       & 0,10                                       & 0,10                                       \\ \hline
  \rowcolor[HTML]{E2EFDA} 
  600cm                                             & lx   Messreihe Innen 4 in Lux                                   & 84                                & 36                                & 12                                & 36                                & 36                                & \cellcolor[HTML]{A9D08E}11950         & D   Messreihe Innen 4 in \%                                     & 0,70                                      & 0,30                                       & 0,10                                       & 0,30                                       & 0,30                                       \\ \hline
  \rowcolor[HTML]{C6E0B4} 
  700cm                                             & lx   Messreihe Innen 5 in Lux                                    & 0                                 & 72                                & 12                                & 0                                 & 0                                 & \cellcolor[HTML]{A9D08E}10770         & D  Messreihe Innen 5 in \%                                      & 0,00                                      & 0,67                                       & 0,11                                       & 0,00                                       & 0,00                                       \\ \hline
  \rowcolor[HTML]{E2EFDA} 
  800cm                                             & lx   Messreihe Innen 6 in Lux                                   & 60                                & 12                                & 0                                 & 12                                & 0                                 & \cellcolor[HTML]{A9D08E}10770         & D   Messreihe Innen 6 in \%                                     & 0,56                                      & 0,11                                       & 0,00                                       & 0,11                                       & 0,00                                       \\ \hline
  \end{tabular}%
  }
  \end{table}

Die Messwerte sind in \autoref{tab:170723_Messung1} eingetragen.
\begin{figure}[H]
\centering
\begin{subfigure}[c]{0.5\textwidth}
      \includegraphics[width=\textwidth]{TD_M1.png}
      \caption{Tageslichtkoeffizient Messung 1}
      \label{fig:Tageslichtkoeffizient Messung 1}
\end{subfigure}
\hfill
\begin{subfigure}[c]{0.45\textwidth}
      \includegraphics[width=\textwidth]{TD_M1_D.png}
      \caption{Tageslichtkoeffizient Messung 1}
      \label{fig:Tageslichtkoeffizient Messung 1 Draufsicht}
\end{subfigure}
\label{fig:Messung 1 Tageslichtkoeffizient}
\end{figure}
In \autoref{sub@fig:Tageslichtkoeffizient Messung 1} und \autoref{sub@fig:Tageslichtkoeffizient Messung 1 Draufsicht} ist die unterschiedliche Bauweise der Fenster und deren Auswirkung auf den Tageslichtkoeffizienten erkennbar. Durch das linke Fenster fällt im Vergleich zum rechten Fenster mehr Licht auf Position 1 und 2 der ersten Messung. Die Lamellen im rechten Fenster lenken das Licht an die Decke um, wodurch weniger Licht direkt auf die Sensoren fällt.
\\Die Gleichmäßigkeitsverteilung $U_0$ des Tageslichts wird mithilfe von \autoref{eq:170723_U_0} berechnet:
\begin{equation}
  U_0 = \frac{E_{min}}{\bar E}
  \label{eq:170723_U_0}
\end{equation}

Für Messung 1 müssen Ausreißer zur Berechnung entfernt werden. Alle Werte gleich null, werden nicht für den minimalen Wert berücksichtigt.
$$U_0 = \frac{12\frac{lux}{m^2}}{94\frac{lux}{m^2}} \approx 12,77\%$$

Analog zu \autoref{eq:170723_U_0} wird die Gleichmäßigkeit des Tageslichtkoeffizienten mithilfe von \autoref{eq:170723_G} berechnet

\begin{equation}
  G = \frac{D_{min}}{\bar D}
  \label{eq:170723_G}
\end{equation}
$$G = \frac{0,1}{1,06} \approx 9,5\%$$
\newpage
\subsection{Tageslichtkoeffizient Messung 2}
\begin{table}[H]
  \centering
  \caption{Messreihe 2, äußere Bestrahlungsstärke, Tageslichtkoeffizient}
  \label{tab:170723_Messung2}
  \resizebox{\columnwidth}{!}{%
  \begin{tabular}{|l|l|l|l|l|l|l|l|l|l|l|l|l|l|}
  \hline
  \rowcolor[HTML]{70AD47} 
  {\color[HTML]{000000} \textbf{Abstand Fenster}} & {\color[HTML]{000000} \textbf{Sensor Position Raumtiefe}} & {\color[HTML]{000000} \textbf{1}} & {\color[HTML]{000000} \textbf{2}} & {\color[HTML]{000000} \textbf{3}} & {\color[HTML]{000000} \textbf{4}} & {\color[HTML]{000000} \textbf{5}} & {\color[HTML]{000000} \textbf{$E_A$ in Lux}} & {\color[HTML]{000000} \textbf{Sensor Position Raumtiefe2}} & {\color[HTML]{000000} \textbf{Position1}} & {\color[HTML]{000000} \textbf{Position 2}} & {\color[HTML]{000000} \textbf{Position 3}} & {\color[HTML]{000000} \textbf{Position 4}} & {\color[HTML]{000000} \textbf{Position 5}} \\ \hline
  \rowcolor[HTML]{C6E0B4} 
  300cm                                             & lx   Messreihe Innen 1 in Lux                                   & 576                               & 528                               & 168                               & 48                                & 432                               & \cellcolor[HTML]{A9D08E}10930                                 & D   Messreihe Innen 1 in \%                                     & 5,27                                      & 4,83                                       & 1,54                                       & 0,44                                       & 3,95                                       \\ \hline
  \rowcolor[HTML]{E2EFDA} 
  cm                                             & lx   Messreihe Innen 2 in Lux                                   & 300                               & 264                               & 252                               & 348                               & 252                               & \cellcolor[HTML]{A9D08E}10930                                 & D   Messreihe Innen 2 in \%                                     & 2,74                                      & 2,42                                       & 2,31                                       & 3,18                                       & 2,31                                       \\ \hline
  \rowcolor[HTML]{C6E0B4} 
  500cm                                             & lx   Messreihe Innen 3 in Lux                                   & 24                                & 168                               & 132                               & 120                               & 84                                & \cellcolor[HTML]{A9D08E}11810                                 & D   Messreihe Innen 3 in \%                                     & 0,20                                      & 1,42                                       & 1,12                                       & 1,02                                       & 0,71                                       \\ \hline
  \rowcolor[HTML]{E2EFDA} 
  600cm                                             & lx   Messreihe Innen 4 in Lux                                   & 96                                & 72                                & 60                                & 96                                & 72                                & \cellcolor[HTML]{A9D08E}11810                                 & D   Messreihe Innen 4 in \%                                     & 0,81                                      & 0,61                                       & 0,51                                       & 0,81                                       & 0,61                                       \\ \hline
  \rowcolor[HTML]{C6E0B4} 
  700cm                                             & lx  Messreihe Innen 5 in Lux                                    & 1                                 & 96                                & 36                                & 12                                & 12                                & \cellcolor[HTML]{A9D08E}10340                                 & D  Messreihe Innen 5 in \%                                      & 0,01                                      & 0,93                                       & 0,35                                       & 0,12                                       & 0,12                                       \\ \hline
  \rowcolor[HTML]{E2EFDA} 
  800cm                                             & lx   Messreihe Innen 6 in Lux                                   & 72                                & 36                                & 12                                & 36                                & 36                                & \cellcolor[HTML]{A9D08E}10340                                 & D   Messreihe Innen 6 in \%                                     & 0,70                                      & 0,35                                       & 0,12                                       & 0,35                                       & 0,35                                       \\ \hline
  \end{tabular}%
  }
  \end{table}
  Die Messwerte sind in \autoref{tab:170723_Messung2} eingetragen.
\begin{figure}[H]
  \centering
  \begin{subfigure}[c]{0.5\textwidth}
      \includegraphics[width=\textwidth]{TD_M2.png}
      \caption{Tageslichtkoeffizient Messung 2}
      \label{fig:Tageslichtkoeffizient Messung 2}
  \end{subfigure}
  \hfill
  \begin{subfigure}[c]{0.4\textwidth}
      \includegraphics[width=\textwidth]{TD_M2_D.png}
      \caption{Tageslichtkoeffizient Messung 2}
      \label{fig:Tageslichtkoeffizient Messung 2 Draufsicht}
  \end{subfigure}
  \label{fig:Messung 2 Tageslichtkoeffizient}
  \end{figure}
  In \autoref{sub@fig:Tageslichtkoeffizient Messung 2} und \autoref{sub@fig:Tageslichtkoeffizient Messung 2 Draufsicht} ist die räumliche Verteilung des Tageslichtkoeffizienten für Messung zwei aufgetragen.
  Im Vergleich zur ersten Messreihe fällt durch das geöffnete Fenster ähnlich viel Licht ein wie durch das geschlossene. Die Verschattung des Sensors auf Position 4 durch einen Stuhl ist gut erkennbar.
  Für Messung 2 müssen Ausreißer zur Berechnung entfernt werden. Alle Werte $<0,01$, werden nicht für den minimalen Wert berücksichtigt.
  $$U_0 = \frac{12\frac{lux}{m^2}}{153,1\frac{lux}{m^2}}\approx 7,84 \%$$
  Die Gleichmäßigkeit des Tageslichtskoeffizient für Messung 2 wird analog zu Messung 1 berechnet.
  $$G =\frac{0,12}{1,34}\approx 8,66\% $$
\newpage
  \subsection{Tageslichtkoeffizient Messung 3}  
  \begin{table}[H]
    \centering
    \caption{Messreihe 3, äußere Bestrahlungsstärke, Tageslichtkoeffizient}
    \label{tab:170723_Messung3}
    \resizebox{\columnwidth}{!}{%
    \begin{tabular}{|l|l|l|l|l|l|l|l|l|l|l|l|l|l|}
    \hline
    \rowcolor[HTML]{70AD47} 
    {\color[HTML]{000000} \textbf{Abstand Fenster}} & {\color[HTML]{000000} \textbf{Sensor Position Raumtiefe}} & {\color[HTML]{000000} \textbf{1}} & {\color[HTML]{000000} \textbf{2}} & {\color[HTML]{000000} \textbf{3}} & {\color[HTML]{000000} \textbf{4}} & {\color[HTML]{000000} \textbf{5}} & {\color[HTML]{000000} \textbf{$E_A$ in Lux}} & {\color[HTML]{000000} \textbf{Sensor Position Raumtiefe2}} & {\color[HTML]{000000} \textbf{Position1}} & {\color[HTML]{000000} \textbf{Position 2}} & {\color[HTML]{000000} \textbf{Position 3}} & {\color[HTML]{000000} \textbf{Position 4}} & {\color[HTML]{000000} \textbf{Position 5}} \\ \hline
    \rowcolor[HTML]{C6E0B4} 
    300cm                                             & lx   Messreihe Innen 1 in Lux                                   & 1668                              & 1512                              & 1032                              & 1092                              & 936                               & \cellcolor[HTML]{A9D08E}{\color[HTML]{000000} 11660}          & D   Messreihe Innen 1 in \%                                     & 14,31                                     & 12,97                                      & 8,85                                       & 9,37                                       & 8,03                                       \\ \hline
    \rowcolor[HTML]{E2EFDA} 
    400cm                                             & lx   Messreihe Innen 2 in Lux                                   & 1188                              & 1248                              & 1032                              & 972                               & 927                               & \cellcolor[HTML]{A9D08E}{\color[HTML]{000000} 11660}          & D   Messreihe Innen 2 in \%                                    & 10,19                                     & 10,70                                      & 8,85                                       & 8,34                                       & 7,95                                       \\ \hline
    \rowcolor[HTML]{C6E0B4} 
    500cm                                             & lx   Messreihe Innen 3 in Lux                                   & 1176                              & 1152                              & 1032                              & 1188                              & 924                               & \cellcolor[HTML]{A9D08E}{\color[HTML]{000000} 11540}          & D   Messreihe Innen 3 in \%                                     & 10,19                                     & 9,98                                       & 8,94                                       & 10,29                                      & 8,01                                       \\ \hline
    \rowcolor[HTML]{E2EFDA} 
    600cm                                             & lx   Messreihe Innen 4 in Lux                                   & 1212                              & 1308                              & 1092                              & 1200                              & 1152                              & \cellcolor[HTML]{A9D08E}{\color[HTML]{000000} 11540}          & D   Messreihe Innen 4 in \%                                     & 10,50                                     & 11,33                                      & 9,46                                       & 10,40                                      & 9,98                                       \\ \hline
    \rowcolor[HTML]{C6E0B4} 
    700cm                                             & lx  Messreihe Innen 5 in Lux                                    & 780                               & 1020                              & 936                               & 1008                              & 936                               & \cellcolor[HTML]{A9D08E}{\color[HTML]{000000} 9370}           & D  Messreihe Innen 5 in \%                                      & 8,32                                      & 10,89                                      & 9,99                                       & 10,76                                      & 9,99                                       \\ \hline
    \rowcolor[HTML]{E2EFDA} 
    800cm                                             & lx   Messreihe Innen 6 in Lux                                   & 876                               & 1248                              & 1020                              & 960                               & 1140                              & \cellcolor[HTML]{A9D08E}{\color[HTML]{000000} 9370}           & D   Messreihe Innen 6 in \%                                     & 9,35                                      & 13,32                                      & 10,89                                      & 10,25                                      & 12,17                                      \\ \hline
    \end{tabular}%
    }
    \end{table}
    Die Messwerte sind in \autoref{tab:170723_Messung3} eingetragen.
  \begin{figure}[H]
  \centering
  \begin{subfigure}[c]{0.5\textwidth}
      \includegraphics[width=\textwidth]{TD_M3.png}
      \caption{Tageslichtkoeffizient Messung 3}
      \label{fig:Tageslichtkoeffizient Messung 3}
  \end{subfigure}
  \hfill
  \begin{subfigure}[c]{0.45\textwidth}
      \includegraphics[width=\textwidth]{TD_M3_D.png}
      \caption{Tageslichtkoeffizient Messung 3}
      \label{fig:Tageslichtkoeffizient Messung 3 Draufsicht}
  \end{subfigure}
  \label{fig:Messung 3 Tageslichtkoeffizient}
  \end{figure}
  In \autoref{sub@fig:Tageslichtkoeffizient Messung 3} und \autoref{sub@fig:Tageslichtkoeffizient Messung 3 Draufsicht} ist im Vergleich zu Messreihe eins und zwei klar die Auswirkung des geschlossenen Fensters und des Kunstlichts ausgeprägt.
  Die Tageslichtkoeffizienten steigen durch das Kunstlicht um mehr als 5\% im gesamten Raum.  Für die erste Messung sind die Werte auf Positon drei, vier und fünf geringer als im restlichen Raum. Die Lamellen im rechten Fenster mindern den Lichteinfall in diesem Bereich stark. Die Rückstrahlung des Kunstlichts von den Wänden ist im hinteren Bereich des Raumes erkennbar. Auf Position 1 bei 7m Abstand zum Fenster kommt es zu leicht niedrigeren Werten, was entweder an einem Messfehler oder an der angebrachten Tafel liegen könnte.
Für Messung 3 müssen keine Anpassungen vorgenommen werden um aussagekräftige Werte für Gleichmäßigkeitsverteilung von $U_0$ zu erhalten.
  $$U_0 = \frac{780\frac{lux}{m^2}}{1089,9\frac{lux}{m^2}}\approx 70,98\%$$
  Die Gleichmäßigkeit des Tageslichtskoeffizient für Messung 4 wird analog zu Messung 1 berechnet.
  $$G = \frac{7,95}{10,15}\approx 78,13\%$$
\newpage
\subsection{Interpretation der Ergebnisse}
Ohne Beleuchtung ist die Gleichmäßigkeit des Tageslichts $U_{0}$ relativ gering und liegt bei ca. 8-13\%, was nicht ausreichend ist um bei Bewölkung die Vorgaben für Arbeitsplätze zu erfüllen.
Mit Beleuchtung steigt dieser Wert auf $\approx 71\%$ an, womit der Raum gleichmäßig ausgeleuchtet ist.

Für die ersten zwei Messreihen zeigen die Daten, dass ohne Zusatzbeleuchtung keine ausreichende Helligkeit für den Arbeitsplatz besteht. Die Beleuchtungstärke ist in den ersten beiden Messungen deutlich unter dem Mindestwert von 500lux, wohingegen diese in Messreihe drei um den Faktor zwei den Sollwert überschreitet. 
Die Gleichmäßigkeit des Tageslichtskoeffizienten ist abhängig von der Bestrahlungsstärke und verhält sich analog. Wohingegen die ersten zwei Messungen für $\bar D$ unter 2\% liegen, steigt die der durchschnittliche Tageslichtkoeffizient für Messreihe 3 auf 10\%. Werte über zwei sind befriedigend und das Mindestmaß für Arbeitsplätze.
Für Messung drei ist dies mit 10,15\% deutlich erfüllt.
Der G für die dritte Messung liegt bei 78\%, was einen sehr guten Wert darstellt.
Durch die installierte Beleuchtung wurde eine ausreichende Helligkeit am Arbeitsplatz gewährleistet.

