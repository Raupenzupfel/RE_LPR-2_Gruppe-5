\section{Auswertung}
\label{sec:Auswertung}
\begin{figure}[H]
    \centering
    \includegraphics[width=\textwidth]{plot1}
    \caption{Bestrahlungsstärke je nach Entfernung aufgeteilt in links, mitte und rechts im Raum bei geschlossenem Fenster. Dabei ist die empfohlene  Mindestbeleuchtungsstärke mit 500 Lux eingetragen.}
    \label{fig:plot1_28062023}
  \end{figure}
  \newpage
  \begin{figure}[H]
    \centering
    \includegraphics[width=\textwidth]{plot2}
    \caption{Bestrahlungsstärke je nach Entfernung aufgeteilt in links, mitte und rechts im Raum bei offenem Fenster. Dabei ist die empfohlene  Mindestbeleuchtungsstärke mit 500 Lux eingetragen. }
    \label{fig:plot2_28062023}
  \end{figure}
  \newpage
  \begin{figure}[H]
    \centering
    \includegraphics[width=\textwidth]{plot3}
    \caption{Bestrahlungsstärke je nach Entfernung aufgeteilt in links, mitte und rechts im Raum bei geschlossenem Fenster mit Kunstlicht. Dabei ist die empfohlene  Mindestbeleuchtungsstärke mit 500 Lux eingetragen.}
    \label{fig:plot3_28062023}
  \end{figure}
  Die Schlussfolgerungen aus der Auswertung der vorliegenden Daten sind wie folgt: Die äußeren Messwerte wurden gemittelt, um eine durchschnittliche Bestrahlungsstärke zu erhalten, während der mittlere Wert keinen Durchschnitt darstellt, sondern einen spezifischen Messwert repräsentiert.
 Der Sollwert der Beleuchtungsstärke für Arbeitsplätze ist auf 500 Lux festgelegt. Messreihe 1 bei geschlossenem Fenster und Messreihe 2 bei offenem Fenser und ausgeschaltetem Licht erfüllen die Anforderung für Arbeitsplätze nicht. Einzig bei eingeschalteter zusätzlicher Beleuchtung werden die 500 Lux erreicht.
Außerdem lässt sich feststellen, dass die Beleuchtungsstärke im betrachteten Raum von Fenster zu Tür hin abnimmt. Diese Abnahme zeigt sich in allen drei betrachteten Szenarien.
Des Weiteren zeigt sich, dass das Öffnen des Fensters, insbesondere wenn Prismenglas verwendet wird, zu einer signifikanten Steigerung der Beleuchtungsstärke führt. In der rechten Raumhälfte ist eine deutliche Zunahme zu beobachten, während in der linken Raumhälfte eine moderate Steigerung festzustellen ist.
Die Verwendung von künstlichem Licht trägt dazu bei, die Beleuchtungsverhältnisse im Raum zu stabilisieren. Allerdings führt dies auch zu erhöhten Beleuchtungsverhältnissen in der Nähe des Fensters. Um den Energieverbrauch zu senken und eine ausgewogenere Beleuchtungsstärke zwischen künstlichem Licht und Tageslicht zu erzielen, könnte eine Dimmung der Leuchten im Fensterbereich in Betracht gezogen werden.
Abschließend ist anzumerken, dass Tageslichtquotienten üblicherweise in Arbeitsbereichen bestimmt werden. Diese Quoten sollten idealerweise zwischen 1\% und 10\% liegen, um eine angemessene Beleuchtung für die durchzuführenden Arbeiten zu gewährleisten.
\newpage
\subsection{Tageslichtkoeffizient Messung 1}
\begin{table}[H]
  \centering
  \caption{Messreihe 1, äußere Bestrahlungsstärke, Tageslichtkoeffizient}
  \label{tab:170723_Messung1}
  \small
  \resizebox{\columnwidth}{!}{%
  \begin{tabular}{|l|l|l|l|l|l|l|l|l|l|l|l|l|l|}
  \hline
  \rowcolor[HTML]{70AD47} 
  {\color[HTML]{343434} \textbf{Abstand Fenster}} & {\color[HTML]{343434} \textbf{Sensor Position Raumtiefe}} & {\color[HTML]{343434} \textbf{1}} & {\color[HTML]{343434} \textbf{2}} & {\color[HTML]{343434} \textbf{3}} & {\color[HTML]{343434} \textbf{4}} & {\color[HTML]{343434} \textbf{5}} & {\color[HTML]{343434} \textbf{20000}} & {\color[HTML]{343434} \textbf{Sensor Position Raumtiefe2}} & {\color[HTML]{343434} \textbf{Position1}} & {\color[HTML]{343434} \textbf{Position 2}} & {\color[HTML]{343434} \textbf{Position 3}} & {\color[HTML]{343434} \textbf{Position 4}} & {\color[HTML]{343434} \textbf{Position 5}} \\ \hline
  \rowcolor[HTML]{C6E0B4} 
  300                                             & lx   Messreihe Innen 1                                    & 588                               & 552                               & 156                               & 0                                 & 24                                & \cellcolor[HTML]{A9D08E}10980         & D   Messreihe Innen 1                                      & 5,36                                      & 5,03                                       & 1,42                                       & 0,00                                       & 0,22                                       \\ \hline
  \rowcolor[HTML]{E2EFDA} 
  400                                             & lx   Messreihe Innen 2                                    & 300                               & 264                               & 156                               & 96                                & 60                                & \cellcolor[HTML]{A9D08E}10980         & D   Messreihe Innen 2                                      & 2,73                                      & 2,40                                       & 1,42                                       & 0,87                                       & 0,55                                       \\ \hline
  \rowcolor[HTML]{C6E0B4} 
  500                                             & lx   Messreihe Innen 3                                    & 36                                & 132                               & 60                                & 12                                & 12                                & \cellcolor[HTML]{A9D08E}11950         & D   Messreihe Innen 3                                      & 0,30                                      & 1,10                                       & 0,50                                       & 0,10                                       & 0,10                                       \\ \hline
  \rowcolor[HTML]{E2EFDA} 
  600                                             & lx   Messreihe Innen 4                                    & 84                                & 36                                & 12                                & 36                                & 36                                & \cellcolor[HTML]{A9D08E}11950         & D   Messreihe Innen 4                                      & 0,70                                      & 0,30                                       & 0,10                                       & 0,30                                       & 0,30                                       \\ \hline
  \rowcolor[HTML]{C6E0B4} 
  700                                             & lx  Messreihe Innen 5                                     & 0                                 & 72                                & 12                                & 0                                 & 0                                 & \cellcolor[HTML]{A9D08E}10770         & D  Messreihe Innen 5                                       & 0,00                                      & 0,67                                       & 0,11                                       & 0,00                                       & 0,00                                       \\ \hline
  \rowcolor[HTML]{E2EFDA} 
  800                                             & lx   Messreihe Innen 6                                    & 60                                & 12                                & 0                                 & 12                                & 0                                 & \cellcolor[HTML]{A9D08E}10770         & D   Messreihe Innen 6                                      & 0,56                                      & 0,11                                       & 0,00                                       & 0,11                                       & 0,00                                       \\ \hline
  \end{tabular}%
  }
  \end{table}

Die Messwerte sind in \autoref{tab:170723_Messung1} eingetragen.
\begin{figure}[H]
\centering
\begin{subfigure}[c]{0.5\textwidth}
      \includegraphics[width=\textwidth]{TD_M1.png}
      \caption{Tageslichtkoeffizient Messung 1}
      \label{fig:Tageslichtkoeffizient Messung 1}
\end{subfigure}
\hfill
\begin{subfigure}[c]{0.45\textwidth}
      \includegraphics[width=\textwidth]{TD_M1_D.png}
      \caption{Tageslichtkoeffizient Messung 1}
      \label{fig:Tageslichtkoeffizient Messung 1 Draufsicht}
\end{subfigure}
\label{fig:Messung 1 Tageslichtkoeffizient}
\end{figure}

Die Gleichmäßigkeitsverteilung $U_0$ des Tageslichts wird mithilfe von \autoref{eq:170723_U_0} berechnet:
\begin{equation}
  U_0 = \frac{E_{min}}{\bar E}
  \label{eq:170723_U_0}
\end{equation}

Für Messung 1 müssen Ausreißer zur Berechnung entfernt werden. Alle Werte gleich null, werden nicht für den minimalen Wert berücksichtigt.
$$U_0 = \frac{12\frac{lux}{m^2}}{94\frac{lux}{m^2}} \approx 12,77\%$$

Analog zu \autoref{eq:170723_U_0} wird die Gleichmäßigkeit des Tageslichtkoeffizienten mithilfe von \autoref{eq:170723_G} berechnet

\begin{equation}
  G = \frac{D_{min}}{\bar D}
  \label{eq:170723_G}
\end{equation}
$$G = \frac{0,1}{1,06} \approx 9,5\%$$
\newpage
\subsection{Tageslichtkoeffizient Messung 2}
\begin{table}[]
  \centering
  \caption{Messreihe 2, äußere Bestrahlungsstärke, Tageslichtkoeffizient}
  \label{tab:170723_Messung2}
  \resizebox{\columnwidth}{!}{%
  \begin{tabular}{|l|l|l|l|l|l|l|l|l|l|l|l|l|l|}
  \hline
  \rowcolor[HTML]{70AD47} 
  {\color[HTML]{000000} \textbf{Abstand Fenster}} & {\color[HTML]{000000} \textbf{Sensor Position Raumtiefe}} & {\color[HTML]{000000} \textbf{1}} & {\color[HTML]{000000} \textbf{2}} & {\color[HTML]{000000} \textbf{3}} & {\color[HTML]{000000} \textbf{4}} & {\color[HTML]{000000} \textbf{5}} & \cellcolor[HTML]{A9D08E}{\color[HTML]{000000} \textbf{20000}} & {\color[HTML]{000000} \textbf{Sensor Position Raumtiefe2}} & {\color[HTML]{000000} \textbf{Position1}} & {\color[HTML]{000000} \textbf{Position 2}} & {\color[HTML]{000000} \textbf{Position 3}} & {\color[HTML]{000000} \textbf{Position 4}} & {\color[HTML]{000000} \textbf{Position 5}} \\ \hline
  \rowcolor[HTML]{C6E0B4} 
  300                                             & lx   Messreihe Innen 1                                    & 576                               & 528                               & 168                               & 48                                & 432                               & \cellcolor[HTML]{A9D08E}10930                                 & D   Messreihe Innen 1                                      & 5,27                                      & 4,83                                       & 1,54                                       & 0,44                                       & 3,95                                       \\ \hline
  \rowcolor[HTML]{E2EFDA} 
  400                                             & lx   Messreihe Innen 2                                    & 300                               & 264                               & 252                               & 348                               & 252                               & \cellcolor[HTML]{A9D08E}10930                                 & D   Messreihe Innen 2                                      & 2,74                                      & 2,42                                       & 2,31                                       & 3,18                                       & 2,31                                       \\ \hline
  \rowcolor[HTML]{C6E0B4} 
  500                                             & lx   Messreihe Innen 3                                    & 24                                & 168                               & 132                               & 120                               & 84                                & \cellcolor[HTML]{A9D08E}11810                                 & D   Messreihe Innen 3                                      & 0,20                                      & 1,42                                       & 1,12                                       & 1,02                                       & 0,71                                       \\ \hline
  \rowcolor[HTML]{E2EFDA} 
  600                                             & lx   Messreihe Innen 4                                    & 96                                & 72                                & 60                                & 96                                & 72                                & \cellcolor[HTML]{A9D08E}11810                                 & D   Messreihe Innen 4                                      & 0,81                                      & 0,61                                       & 0,51                                       & 0,81                                       & 0,61                                       \\ \hline
  \rowcolor[HTML]{C6E0B4} 
  700                                             & lx  Messreihe Innen 5                                     & 1                                 & 96                                & 36                                & 12                                & 12                                & \cellcolor[HTML]{A9D08E}10340                                 & D  Messreihe Innen 5                                       & 0,01                                      & 0,93                                       & 0,35                                       & 0,12                                       & 0,12                                       \\ \hline
  \rowcolor[HTML]{E2EFDA} 
  800                                             & lx   Messreihe Innen 6                                    & 72                                & 36                                & 12                                & 36                                & 36                                & \cellcolor[HTML]{A9D08E}10340                                 & D   Messreihe Innen 6                                      & 0,70                                      & 0,35                                       & 0,12                                       & 0,35                                       & 0,35                                       \\ \hline
  \end{tabular}%
  }
  \end{table}
\begin{figure}[H]
  \centering
  \begin{subfigure}[c]{0.5\textwidth}
      \includegraphics[width=\textwidth]{TD_M2.png}
      \caption{Tageslichtkoeffizient Messung 2}
      \label{fig:Tageslichtkoeffizient Messung 2}
  \end{subfigure}
  \hfill
  \begin{subfigure}[c]{0.4\textwidth}
      \includegraphics[width=\textwidth]{TD_M2_D.png}
      \caption{Tageslichtkoeffizient Messung 2}
      \label{fig:Tageslichtkoeffizient Messung 2 Draufsicht}
  \end{subfigure}
  \label{fig:Messung 2 Tageslichtkoeffizient}
  \end{figure}

  $$U_0 = \frac{12\frac{lux}{m^2}}{153,1\frac{lux}{m^2}}\approx 7,84 \%$$
  $$G =\frac{0,12}{1,34}\approx 8,66\% $$
\newpage
  \subsection{Tageslichtkoeffizient Messung 3}  
  \begin{table}[H]
    \centering
    \caption{Messreihe 3, äußere Bestrahlungsstärke, Tageslichtkoeffizient}
    \label{tab:170723_Messung3}
    \resizebox{\columnwidth}{!}{%
    \begin{tabular}{|l|l|l|l|l|l|l|l|l|l|l|l|l|l|}
    \hline
    \rowcolor[HTML]{70AD47} 
    {\color[HTML]{000000} \textbf{Abstand Fenster}} & {\color[HTML]{000000} \textbf{Sensor Position Raumtiefe}} & {\color[HTML]{000000} \textbf{1}} & {\color[HTML]{000000} \textbf{2}} & {\color[HTML]{000000} \textbf{3}} & {\color[HTML]{000000} \textbf{4}} & {\color[HTML]{000000} \textbf{5}} & \cellcolor[HTML]{A9D08E}{\color[HTML]{000000} \textbf{20000}} & {\color[HTML]{000000} \textbf{Sensor Position Raumtiefe2}} & {\color[HTML]{000000} \textbf{Position1}} & {\color[HTML]{000000} \textbf{Position 2}} & {\color[HTML]{000000} \textbf{Position 3}} & {\color[HTML]{000000} \textbf{Position 4}} & {\color[HTML]{000000} \textbf{Position 5}} \\ \hline
    \rowcolor[HTML]{C6E0B4} 
    300                                             & lx   Messreihe Innen 1                                    & 1668                              & 1512                              & 1032                              & 1092                              & 936                               & \cellcolor[HTML]{A9D08E}{\color[HTML]{000000} 11660}          & D   Messreihe Innen 1                                      & 14,31                                     & 12,97                                      & 8,85                                       & 9,37                                       & 8,03                                       \\ \hline
    \rowcolor[HTML]{E2EFDA} 
    400                                             & lx   Messreihe Innen 2                                    & 1188                              & 1248                              & 1032                              & 972                               & 927                               & \cellcolor[HTML]{A9D08E}{\color[HTML]{000000} 11660}          & D   Messreihe Innen 2                                      & 10,19                                     & 10,70                                      & 8,85                                       & 8,34                                       & 7,95                                       \\ \hline
    \rowcolor[HTML]{C6E0B4} 
    500                                             & lx   Messreihe Innen 3                                    & 1176                              & 1152                              & 1032                              & 1188                              & 924                               & \cellcolor[HTML]{A9D08E}{\color[HTML]{000000} 11540}          & D   Messreihe Innen 3                                      & 10,19                                     & 9,98                                       & 8,94                                       & 10,29                                      & 8,01                                       \\ \hline
    \rowcolor[HTML]{E2EFDA} 
    600                                             & lx   Messreihe Innen 4                                    & 1212                              & 1308                              & 1092                              & 1200                              & 1152                              & \cellcolor[HTML]{A9D08E}{\color[HTML]{000000} 11540}          & D   Messreihe Innen 4                                      & 10,50                                     & 11,33                                      & 9,46                                       & 10,40                                      & 9,98                                       \\ \hline
    \rowcolor[HTML]{C6E0B4} 
    700                                             & lx  Messreihe Innen 5                                     & 780                               & 1020                              & 936                               & 1008                              & 936                               & \cellcolor[HTML]{A9D08E}{\color[HTML]{000000} 9370}           & D  Messreihe Innen 5                                       & 8,32                                      & 10,89                                      & 9,99                                       & 10,76                                      & 9,99                                       \\ \hline
    \rowcolor[HTML]{E2EFDA} 
    800                                             & lx   Messreihe Innen 6                                    & 876                               & 1248                              & 1020                              & 960                               & 1140                              & \cellcolor[HTML]{A9D08E}{\color[HTML]{000000} 9370}           & D   Messreihe Innen 6                                      & 9,35                                      & 13,32                                      & 10,89                                      & 10,25                                      & 12,17                                      \\ \hline
    \end{tabular}%
    }
    \end{table}
  \begin{figure}[H]
  \centering
  \begin{subfigure}[c]{0.5\textwidth}
      \includegraphics[width=\textwidth]{TD_M3.png}
      \caption{Tageslichtkoeffizient Messung 3}
      \label{fig:Tageslichtkoeffizient Messung 3}
  \end{subfigure}
  \hfill
  \begin{subfigure}[c]{0.45\textwidth}
      \includegraphics[width=\textwidth]{TD_M3_D.png}
      \caption{Tageslichtkoeffizient Messung 3}
      \label{fig:Tageslichtkoeffizient Messung 3 Draufsicht}
  \end{subfigure}
  \label{fig:Messung 3 Tageslichtkoeffizient}
  \end{figure}

  $$U_0 = \frac{780\frac{lux}{m^2}}{1089,9\frac{lux}{m^2}}\approx 70,98\%$$

  $$G = \frac{7,95}{10,15}\approx 78,13\%$$

  Zunächst lässt sich feststellen, dass die Beleuchtungsstärke im betrachteten Raum von Fenster zu Tür hin abnimmt. Diese Abnahme zeigt sich in allen drei betrachteten Szenarien.

Des Weiteren zeigt sich, dass das Öffnen des Fensters, insbesondere wenn Prismenglas verwendet wird, zu einer signifikanten Steigerung der Beleuchtungsstärke führt. In der rechten Raumhälfte ist eine deutliche Zunahme zu beobachten, während in der linken Raumhälfte eine moderate Steigerung festzustellen ist.

Die Verwendung von künstlichem Licht trägt dazu bei, die Beleuchtungsverhältnisse im Raum zu stabilisieren. Allerdings führt dies auch zu erhöhten Beleuchtungsverhältnissen in der Nähe des Fensters. Um den Energieverbrauch zu senken und eine ausgewogenere Beleuchtungsstärke zwischen künstlichem Licht und Tageslicht zu erzielen, könnte eine Dimmung der Leuchten im Fensterbereich in Betracht gezogen werden.

Abschließend ist anzumerken, dass Tageslichtquotienten üblicherweise in Arbeitsbereichen bestimmt werden. Diese Quoten sollten idealerweise zwischen 1\% und 10\% liegen, um eine angemessene Beleuchtung für die durchzuführenden Arbeiten zu gewährleisten.