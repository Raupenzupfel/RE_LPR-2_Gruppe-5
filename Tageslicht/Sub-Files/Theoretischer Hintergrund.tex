\section{Theoretischer Hintergrund}
Licht hat eine bedeutende Auswirkung auf Menschen, sowohl psychologisch als auch physiologisch. Eine gute Beleuchtung ist sowohl am Arbeitsplatz als auch zu Hause wichtig. Es gibt gesetzliche Mindestanforderungen für Beleuchtungsstärke und Energieeffizienz. Beleuchtungsanlagen in Büro- und Verwaltungsgebäuden können einen großen Anteil am Energieverbrauch ausmachen. Daher ist eine energiebewusste Beleuchtung mit viel Tageslicht ökologisch und ökonomisch sinnvoll.

Die Merkmale einer guten Beleuchtung umfassen: Ausreichende Funktionalität, Emotionalität und Ästhetik, wobei Tageslicht eine wichtige Rolle spielt. Bei der Beleuchtungsplanung ist es wichtig, Einsparpotenziale und die Auswirkungen auf Menschen zu berücksichtigen.

Die Beleuchtungsstärke misst den gesamten Lichtstrom, der auf eine Flächeneinheit einstrahlt und wird in Lux gemessen. Der Tageslichtquotient wird verwendet, um die Tageslichtqualität eines Raumes zu bewerten. Er beschreibt das Verhältnis der Beleuchtungsstärke im Raum zur Beleuchtungsstärke im Freien ohne Hindernisse und bei bedecktem Himmel. Durch den Tageslichtquotienten kann die innere Beleuchtungsstärke abgeschätzt werden, wenn die Außenbeleuchtungsstärke bekannt ist.

Die Berechnung des Tageslichtquotienten umfasst verschiedene Faktoren wie den Anteil des Himmelslichts, die Außenreflexion und die Innenreflexion.

Es ist zu beachten, dass der Tageslichtquotient lediglich eine Schätzung liefert und bei direktem Sonnenlicht und sich ändernden Himmelsbedingungen ungenau sein kann. In der Planungsphase kann er jedoch als nützliches Vergleichsinstrument dienen, insbesondere wenn reproduzierbare Himmelszustände simuliert werden können.