\newpage
\section{Auswertung}
\label{sec:Auswertung}
\subsection{Energiebilanz der Anlage und äußere Leistungszahl}
In diesem Teil werden die Leistungszahlen und die Energiebilanz der Anlage berechnet.

\subsubsection{Prüfen Sie die im Bedienmodul angegebene Heizleistung anhand Vor- und Rücklauftemperatur und Heizdurchfluss}
\label{subsubsec:Heizleistung}

Die Heizleistung lässt sich mit der folgenden Formel berechnen:

\begin{equation}
\dot Q_{Kond}= \dot m \cdot cp_{W} \cdot (T_{V} - T_{R})
\label{eq:110623_Heizleistung}
\end{equation}

Dafür muss vorher der Massenstrom über den Volumenstrom nach folgender Formel bestimmt werden.

\begin{equation}
    \dot{m}= \dot{V}\cdot \rho
\end{equation}

$$\dot{m}= 0,391 \cdot 10^{-3} \cdot 990 \frac{kg}{m^3}=0,38 \frac{m}{s}$$

Mit Hilfe dieses Massenstroms kann die Heizleistung berechnet werden.

$$\dot Q_{Kond}= 0,38 \frac{kg}{s} \cdot 4,2 \frac{kJ}{kg \cdot K} \cdot (48 ^{\circ}C - 43 ^{\circ}C) = 8,14 \frac{kJ}{s}$$

Mit einem Massenstrom von 0,38 $\frac{kg}{s}$, einer Vorlauftemperatur von 48°C und Rücklauftemperatur von 43°C im Kondensator ergibt sich daraus mit Formel \ref{eq:110623_Heizleistung} eine Heizleistung von 8,14 $\frac{kJ}{s}$ bzw $kW$. Wobei auf dem Display eine Heizleistung von 8,3kW angezeit wird, was nur minimal von unserem errechneten Wert abweicht.
\subsubsection{Berechnen Sie die äußere Leistungszahl der Anlage}
Die tatsächliche äußere Leistungszahl berechnet sich wie in Formel \ref{eq:110623_aeußere Leistungszahl_tatsächlich} dargestellt und beträgt 3,54.

\begin{equation}
\epsilon_{Real,Au"sen} = \frac{\dot Q_{Nutz}}{P_{el}}
\label{eq:110623_aeußere Leistungszahl_tatsächlich}
\end{equation}

$$ \epsilon_{Real,Au"sen} = \frac{8,14 kW}{2,3 kW} = 3,54$$

\subsubsection{Berechnen Sie aus der Gesamt-Energiebilanz der Anlage den aus der Umgebung aufgenommenen Wärmestrom im Verdampfer.
Welcher Luftvolumenstrom muss durchgesetzt werden, wenn die Luft um 5 K abgekühlt wird?}
Um den aus der Umgebung aufgenommenen Wärmestrom zu bestimmen, muss die Gesammt-Energiebilanz der Anlage verwendet werden:

\begin{equation}
\dot Q_{Verd}=\dot Q_{Kond}-P_{el}
\label{eq:110623_aeußere Leistungszahl}
\end{equation}

$$\dot Q_{Verd}= 8,14 kW-2,304 kW= 5,84kW $$


Mit der berechneten Heizleistung von 8,14kW und einer elektrischen Leistung von 2304W ergibt sich ein vom Verdampfer aufgenommenen Wärmestrom von 5,84kW. Nun wird der benötigte Luftstrom, unter der Annahme, dass die durchströmende Luft um 5K abgekühlt wird, berechnet.

\begin{equation}
\dot V_{L}=\frac{\dot Q_{verd}}{\rho_{Luft} \cdot cp_{Luft} \cdot \Delta T}
\label{eq:110623_benoetigter_Luftstrom}
\end{equation}

$$\dot V_{L}=\frac{5,84 kW}{ 1,2 \frac{kg}{m^3} \cdot 1 \frac{kJ}{kg \cdot K} \cdot 5K}= 0,973 \frac{m^3}{s}$$


Der benötigte Luftstrom beträgt nach Formel \ref{eq:110623_benoetigter_Luftstrom} 0,973$\frac{m^3}{s}$. 
\subsubsection{Berechnen Sie die äußere Carnot-Leistungszahl der Anlage \texorpdfstring{$\epsilon_{Carnot,a}$}{} anhand der erreichten Mitteltemperatur der Nutzwärme TmN =(TKessel + TRück)/2 und der Umgebungstemperatur.}
Für die äußere Carnot-Leistungszahl der Anlage benötigt man die Umgebungstemperatur $T_{U}=$24°C= 297,15K und die mittlere Temperatur der Nutzwärme.

\begin{equation}
    T_{mN}=\frac{T_{Kessel}+T_{Rück}}{2}
\end{equation}

$$T_{mN}=\frac{51,6^{\circ}C+44,9^{\circ}C}{2}=48,25^{\circ}= 321,4 KC$$

Die äußere Carnot-Leistungszahl berechnet sich dann wie folgt: 

\begin{equation}
\epsilon_{Carnot, Au"sen} = \frac{T_{mN}}{T_{mN}-T_{U}}
\label{eq:110623_aeußere Carnot Leistungszahl}
\end{equation}

$$\epsilon_{Carnot, Au"sen} = \frac{321,4 K}{321,4 K-297,15K} = 13,25$$

Sie beläuft sich dabei auf 13,25.

\subsection{Arbeitsmittelkreislauf und innere Leistungszahlen}
Entsprechend der Werte aus Messreihe 1, welche in \autoref{tab:Arbeitspunkte} zusammengefasst sind, kann das passende
lg(p)-h-Diagramm, mit den Arbeitspunkten 1-4 erstellt werden. Aus diesem Diagramm lassen sich die in \autoref{tab:Arbeitspunkte} ebenfalls
notierten Enthalpien ablesen.

\begin{table}[!h]
    \centering
    \resizebox{\textwidth}{!}{%
    \begin{tabular}{|c|c|c|c|}
    \hline
    \multicolumn{1}{|l|}{Arbeitspunkt} & Bezeichnung                  & \multicolumn{1}{l|}{Temperatur in °C} & \multicolumn{1}{l|}{Enthalpie in kJ/kg} \\ \hline
    1                                  & Temperatur nach Verdampfer   & 18                                    & 425                                     \\ \hline
    2                                  & Heißgastemperatur            & 63                                    & 448                                     \\ \hline
    3                                  & Temperatur nach Verflüssiger & 43                                    & 280                                     \\ \hline
    4                                  & Temperatur vor Verdampfer    & 16                                    & 280                                     \\ \hline
    \end{tabular}%
    }
    \caption{Arbeitspunkte entsprechend Messreihe 1}
    \label{tab:Arbeitspunkte}
    \end{table}

    Zusätzlich kann das obere Druckniveau von 29,5 bar und das unter Druckniveau von rund 13 bar abgelesen werden.
    Des Weiteren ist in \autoref{fig:lgp-h-Diagramm} ersichtlich, dass die Ergebnisse plausibel sind, da der dargestellte Kreisprozess dem erwarteten Bild entspricht.\\

    \begin{figure}[!h]
        \centering
        \includegraphics[width=0.5\textwidth]{Abbildungen/lgp-h-diagramm.jpg}
        \caption{lg(p)-h-Diagramm mit den Arbeitspunkten aus Messreihe 1}
        \label{fig:lgp-h-Diagramm}
    \end{figure}

Mit den Temperaturen aus \autoref{tab:Arbeitspunkte} kann mittels \autoref{eq:Innere Carnot-Leistungszahl} die innere Carnot-Leistungszahl
ermittelt werden. Die Berechnung ergibt dann eine innere Carnot-Leistungszahl von 11,7.

    \begin{equation}
        \epsilon_{Carnot, Innen}=\frac{T_{max}}{T_{max}-T_{min}}=\frac{T_{Kond}}{T_{Kond}-T_{Verd}}
        \label{eq:Innere Carnot-Leistungszahl}
    \end{equation}

$$\epsilon_{Carnot, Innen}=\frac{316,15 K}{316,15 K-291,15 K}=12,6$$

 In \autoref{fig:lgp-h-Diagramm-is} ist zusätzlich der isentrope Arbeitspunkt 2 eingezeichnet und somit die isentrope Enthalpie für diesen Arbeitspunkt von $440 \frac{kJ}{kg}$
 eingezeichnet und abzulesen.
    
 \begin{figure}[!h]
        \centering
        \includegraphics[width=0.5\textwidth]{Abbildungen/lgp-h-diagramm-is.jpg}
        \caption{lg(p)-h-Diagramm mit dem isentropen Arbeitspunkt 2}
        \label{fig:lgp-h-Diagramm-is}
    \end{figure}

    Mittels der nun bekannten isentropen Enthalpie für Arbeitspunkt 2 kann mittels \autoref{eq:Isentropenwirkungsgrad} der Isentropenwirkungsgrad ermittelt werden.
\begin{equation}
    \eta_{is}=\frac{h_{2,is}-h_{1}}{h_{2}-h_{1}}
    \label{eq:Isentropenwirkungsgrad}
\end{equation}

$$\eta_{is}=\frac{433 kJ/kg-425 kJ/kg}{448 kJ/kg-425 kJ/kg}=0,6521$$

Der Isentropenwirkungsgrad für diesen Fall beträgt demnach 0,6521.\\


\subsubsection*{Innere Leistungszahlen}

\begin{equation}
   \epsilon_{VP,i} = \frac{h_{2,is}-h_{pH}}{h_{2,is}-h^{''}_{p0}}
\label{eq:230623_Leistungszahl_VPI}
\end{equation}

Aus der Dampftafel für das Kältemittel R410a wurden die Werte $h"_{p0}=425,46 \frac{kJ}{kg}$ und $h_{pH}= 225,03 \frac{kJ}{kg}$ entnommen und mit den Werten aus \autoref*{Arbeitspunkte} in \autoref{eq:230623_Leistungszahl_VPI} eingesetzt
$$\epsilon_{VP,i} = \frac{h_{2,is}-h_{pH}}{h_{2,is}-h^{''}_{p0}}$$

\begin{equation}
    \epsilon_{real,i}= \frac{h_2-h_3}{h_2-h_1}
\end{equation}

$$\epsilon_{real,i}= \frac{(448-280)\frac{kJ}{kg}}{(448-425)\frac{kJ}{kg}} = 7,30 $$

\subsection{Leistungszahlen und Verlustursachen}

% Leistungszahl auflistung nach finalisierung von Niels und Lukas

$$\epsilon_{Carnot,Außen} = 13,25$$
$$\epsilon_{Carnot,Innen} = 11,7$$
$$\epsilon_{VP,Innen} = x,xx$$
$$\epsilon_{Real,Innen} = x,xx$$
$$\epsilon_{Real,Au"sen} = 3,53$$

Wie ersichtlich ist, gibt $\epsilon_{Carnot,A"sen}$ die maximale Leistungszahl an. 
Dies liegt daran, dass der äußere Carnot-Wirkungsgrad einen idealisierten Prozess darstellt, bei dem Verluste vollständig vernachlässigt werden. 
Er vergleicht lediglich die Temperaturänderung der mittleren Nutzwärme mit der Umgebungstemperatur.

Der äußere Carnot-Wirkungsgrad ist größer als der innere Carnot-Wirkungsgrad $\epsilon_{Carnot,Innen}$, da sich der innere auf die Kondensations- und Verdampfungstemperatur stützt. 
Um die Wärme auf den Prozess zu übertragen und wieder abzugeben, sind Temperaturunterschiede an den beiden Wärmetauschern erforderlich. 
Bei der Wärmepumpe erfolgt die Wärmeübertragung am Verdampfer und am Kondensator. 
Diese Verluste der Wärmeübertragung werden beim inneren Carnot-Wirkungsgrad berücksichtigt.

Die innere Leistungszahl für den Vergleichsprozess $\epsilon_{VP,Innen}$ wurde mithilfe der theoretischen Enthalpien des Kältemittelkreislaufs berechnet.
Dies sind abgelesene Werte aus Diagrammen oder Tabellen und können daher etwas ungenauer sein. 
Der theoretische Carnot-Prozess zur allgemeinen Wärmeübertragung wird nun durch einen realen Kältemittelkreislauf realisiert. 
Das Kältemittel Solkane 410A weist stoffspezifische Enthalpien auf. 
Dadurch entstehen Verluste aufgrund des Stoffs und der tatsächlichen Prozessführung.

Als vorletzte Leistungszahl haben wir die reale innere Leistungszahl $\epsilon_{Real,Innen}$.
Diese ist geringer, da die Übergänge im realen Verlauf nicht isentrop, sondern irreversibel sind. 
Die Druck- und Reibungsverluste einzelner Bauteile werden nun berücksichtigt. 
Der Verdampfer sollte das Arbeitsmittel leicht überhitzen, da es sonst bereits in der Leitung zum Verdichter kondensieren könnte. 
Außerdem findet die Wärmezufuhr nicht wie theoretisch berechnet isobar statt.

Die Verdichtung weist Verluste auf und erfolgt nicht isentrop.

Im Kondensator ist das Arbeitsmittel überhitzt und gibt daher Wärme bei einer höheren Temperatur als der Kondensationstemperatur ab. 
Dadurch entstehen Wärme- und Druckverluste, ähnlich wie im Verdampfer. 
Die Kondensation findet aufgrund der Verluste nicht isobar statt.

Die Drossel arbeitet relativ effizient, aber nicht isenthalp.

Die geringste Leistungszahl weist die äußere Leistungszahl $\epsilon_{Real,Au"sen}$ auf. 
Dieser Wert berücksichtigt den tatsächlich benötigten Nutzwärmestrom und die dafür erforderliche elektrische Verdichterleistung. 
Er umfasst daher alle Verluste im gesamten System.


\subsection{Energiebilanzen Einzelapparate}

\subsubsection{} 
\label{subsec:Massenstrom}

\subsubsection{}
\subsubsection{Der Durchmesser der Heißgasleitung beträgt 16 mm, der der Kondensatleitung 10
mm. Berechnen Sie mithilfe der aus Diagramm bzw. Dampftafel abgelesenen spezifischen Volumina die Strömungsgeschwindigkeiten von Gas und Flüssigkeit.}

Das spezifische Volumen des Gases kann aus dem lg(p)-h Diagramm abgelesen werden zu: $V_{Gas}=0,02 m^3/kg$. Die Dichte, welches der Kehrwert des spezifischen Volumens ist, ergibt sich als $\rho=50 kg/m^3$. 
Mit dem durchschnittlichen Massenstrom aus \autoref{subsec:Massenstrom} können die Strömungsgeschwindigkeiten über den Volumenstrom nach folgenden Gleichungen berechnet werden.

\begin{equation}
    \dot{V}= \frac{\dot{m}}{\rho }
    \label{eq:230620_Volumenstrom}
\end{equation}

\begin{equation}
    \dot{V}= v \cdot \frac{\pi}{4} \cdot d^2
    \label{eq:230620_Volumenstrom2}
\end{equation}

Werden \autoref{eq:230620_Volumenstrom} und \autoref{eq:230620_Volumenstrom2} gleichgesetzt ergibt sich die Strömungsgeschwindigkeit nach \autoref{eq:230620_Strömungsgeschwindigkeiten}.

\begin{equation}
    v = \frac{\dot{m}}{\frac{\pi}{4}\cdot \rho \cdot d^2}
    \label{eq:230620_Strömungsgeschwindigkeiten}
\end{equation}

Am Arbeitspunkt 1 ergibt sich die Strömungsgeschwindigkeit:
$$v=\frac{0,0444 kg/s}{\frac{\pi}{4}\cdot 50 kg/m^3 \cdot (0,016 m)^2}=4,42 m/s $$

Der Druck am Arbeitspunkt 3 kann aus dem lg(p)-h Diagramm als $p=30 bar$ abgelesen werden. Mithilfe dessen lässt sich die Dichte aus der Dampftafel als $\rho=914,34 kg/m^3$ bestimmen.
Die Strömungsgeschwindigkeit der Flüssigkeit am Arbeitspunkt 3 berechnet sich dann zu:
$$v=\frac{0,0444 kg/s}{\frac{\pi}{4}\cdot 914,37 kg/m^3 \cdot (0,01 m)^2}=0,618 m/s $$

\subsubsection{Skizzieren Sie ein T,A-Schaubild des kleinen Gegenstrom-Plattenwärmeübertragers
zwischen internem und externem Heizkreis. Berechnen Sie den erreichten k-Wert,
wenn der Apparat eine Wärmeübertragerfläche eine Fläche von \texorpdfstring{$A = 0,4 m^2$}{} hat.}

\begin{figure}[!h]
    \centering
    \includegraphics[width=0.5\textwidth]{Abbildungen/TA,Diagramm.jpeg}
    \caption{T,A-Schaubild eines Gegenstrom-Plattenwärmeübertragers}
    \label{fig:TA,Diagramm}
\end{figure}



Für die Berechnung des K-Werts muss im Vorfeld $\Delta T_M$ bestimmt werden. Das passiert nach folgender Gleichung.

\begin{equation}
    \Delta T_{MAX}= \Delta T_{HA}-\Delta T_{KE}
    \label{eq:230621_DeltaTMAX}
\end{equation}

$$\Delta T_{MAX}= 41\text{°}C-24 \text{°} C= 17K$$

\begin{equation}
    \Delta T_{MIN}= \Delta T_{HE}-\Delta T_{KA}
    \label{eq:230621_DeltaTMIN}
\end{equation}

$$\Delta T_{MIN}= 54 \text{°} C-51 \text{°} C= 3K$$

\begin{equation}
    \Delta T_{M}= \frac{\Delta T_{MAX}-\Delta T_{MIN}}{ln\frac{\Delta T_{MAX}}{\Delta T_{MIN}}}
    \label{eq:230621_DeltaTM}
\end{equation}

$$\Delta T_M= \frac{17K-3K}{ln\frac{17K}{3K}}= 8,07K$$

Mithilfe der \autoref{eq:230621_Q} und der Heizleistung lässt sich der k-Wert bestimmen.

\begin{equation}
    \dot{Q}=k\cdot A \cdot \Delta T_M
    \label{eq:230621_Q}
\end{equation}

\begin{equation}
    k = \frac{\dot{Q}}{ A \cdot \Delta T_M} 
    \label{eq:230621_k}
\end{equation}

$$k=\frac{8,14 kW}{ 0,4m^2 \cdot 8,07K}=6,3 kW/m^2K$$

\subsubsection{Wie lange dauert es bei konstanter Wärmepumpenleistung, bis ausgehend von einer
konstanten Anfangstemperatur \texorpdfstring{$T_{Sp,U}$}{} der gesamte Speicherinhalt (450 l) aufgeheizt ist?}

Die Wärmemenge die benötigt wird um den Speicher aufzuheizen berechnet sich nach:

\begin{equation}
Q = m \cdot c_p \cdot \Delta T = \rho \cdot V \cdot c_p \cdot \Delta T
\end{equation}

$$Q= 997 kg/m^3 \cdot 0,45 m^3 \cdot 4,18 kJ/kgK (35 \text{°} C-22 \text{°} C)K=24,4 MJ$$

Da die Heizleistung nur eine Energieangabe pro Zeiteinheit ist lässt sich die Zeit durch eine einfache Gleichung bestimmen.

\begin{equation}
    P_{Heiz}= \frac{Q}{t}
\end{equation}

\begin{equation}
 t = \frac{Q}{P_{Heiz}}
\end{equation}

$$ t= \frac{24,4 MJ}{8,14 kW}=2995 s= 49,9 min$$

\subsection{Leistungszahlen und Verlustursachen}

\subsection{Energiebilanzen Einzelapparate}
\subsubsection*{Arbeitsmittelmassenstrom Kondensator/Verdampfer}
\begin{equation}
   \dot m_{AM} = \frac{\dot Q_{Kond}}{(h_3 - h_2)}
   \label{eq:230623_massestrom}
\end{equation}
Mit $\dot Q_{Kondensator}=8,14kW$, $h_2=448\frac{kJ}{kg}$ und $h_3=280\frac{kJ}{kg}$ wird der Massenstrom des Kondensators nach \autoref*{eq:230623_massestrom} berechnet:

$$ \dot m_{AM} = \frac{8,14kW}{|(280\frac{kJ}{kg} - 448\frac{kJ}{kg})|} = 0,04845 \frac{kg}{s} $$

\begin{equation}
  \dot m_{AM} = \frac{\dot Q_{Verd}}{|(h_1-h_4)|}
    \label{eq230623_massestrom_ver}
\end{equation}

Mit $\dot Q_{Verd}=5,84kW$, $h_1=448\frac{kJ}{kg}$ und $h_4=280\frac{kJ}{kg}$ wird der Massenstrom des Kondensators nach \autoref*{eq:230623_massestrom} berechnet:
$$\dot m_{AM} = \frac{5,84 kW}{|(h_1-h_4)|} = 0,04027 \frac{kg}{s}$$

$$\overline{\dot m_{AM}} = \frac{\dot m_{AM_{Kond}}+\dot m_{AM_{Verd}}}{2} = \frac{0,04845 \frac{kg}{s}+0,04027 \frac{kg}{s}}{2} = 0,04436 \frac{kg}{s} $$


